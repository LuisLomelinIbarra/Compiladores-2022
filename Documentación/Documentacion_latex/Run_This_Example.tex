%This file is just a wrapper. Please, edit the files for your chapter in chapters/chapter1/.
%Don't worry, we will put your chapter in the correct place when assemble the book.



\documentclass[krantz1,ChapterTOCs, spanish]{krantz}
\usepackage[spanish]{babel}
\selectlanguage{spanish}
\usepackage[utf8]{inputenc}

\usepackage[T1]{fontenc}
\usepackage{fixltx2e,fix-cm}
\usepackage{amssymb}
\usepackage{amsmath}
\usepackage{graphicx}
\usepackage{subfigure}
\usepackage{makeidx}
\usepackage{multicol}
\usepackage{listings}
\usepackage{placeins}
\usepackage{graphicx}
\usepackage{minted}

%\usepackage[dvips]{hyperref}

\frenchspacing
\tolerance=5000

\makeindex

\include{frontmatter/preamble} %place custom commands and macros here

\begin{document}

\frontmatter

\title{IntroProg el lenguaje éstadistico para aprender a programar 
%in Socio-Environmental Systems, Public Health, and Insurance\\
%{\Large(Applied Environmental Statistics Series)}
}
\author{Luis Fernando Lomelín Ibarra}

\maketitle

\cleardoublepage
\thispagestyle{empty}
\vspace*{\stretch{1}}
\begin{center}
\Large\itshape
Gracias a mi familia. A mi mamá y a mi hermano por apoyarme mucho durante todo este tiempo. Sin ellos realmente no sabría en donde estaría ahora.\\
Gracias a mis amigos. Por estar allí en los momentos buenos y malos. Por hacer la vida más interesante\\
Gracias a mis profesores. Por pasar su conocimiento con paciencia y entusiasmo, y por darme la oportunidad de hacer grandes cosas

\end{center}
\vspace{\stretch{2}}
\cleardoublepage
\setcounter{page}{7} %previous pages will be reserved for frontmatter to be added in later.
\tableofcontents
%\include{frontmatter/foreword}
%\include{frontmatter/preface}
%\listoffigures
%\listoftables
%\include{frontmatter/contributor}
%\include{frontmatter/symbollist}

\mainmatter

\part{Descripción del proyecto y lenguaje}
\chapter{Descripción del proyecto}

Una breve descripción del proyecto, describiendo su alcance al igual que su propósito.

\section{Propósito y Alcance del Proyecto}\label{intro}
El propósito de este lenguaje es ser un lenguaje introductorio a la lógica de programación para personas que no están muy familiarizadas con la programación y que no tienen una buena comprensión del inglés. Para lograr esto el lenguaje fue diseñado a ser similar a C/C++ en estructura, pero se utilizan palabras clave en español para facilitar el aprendizaje de programación para las personas que no tengan un buen dominio en el lenguaje. El objetivo del lenguaje es que sea un punto de entrada para chicos en prepa o alumnos de universidad a aventurarse a lenguajes con sintaxis similar a C/C++ pero que no tienen una gran comprensión del inglés. \\ \\
Con esto en mente el lenguaje busca ser una herramienta básica de programación que, aunque sea restrictiva a comparación de otros lenguajes, permita que el usuario experimente con funciones de estadística y con la lógica de programación para crear pequeños programas con los cuales pueden explorar temas de programación, ciencia de datos y estadística.


\section{Análisis de Requerimientos y descripción de los principales Test Cases}

Con el objetivo y alcance en definido, se plantearon los siguientes requerimientos que se deben de cumplir con el lenguaje.


\begin{itemize}
    \item Crear palabras clave análogas o similares a las palabras clave del lenguaje de C/C++ en español
    \item Generar estructuras similares a las de C/C++ para estatutos como if y while, pero que sean legibles en español
    \item Permite el uso de estructuras de decisión (if, if else)
    \item Permite el uso de estructuras cíclicas (for, while, do while)
    \item Realizar operaciones matemáticas aritméticas sencillas como suma, restas, multiplicaciones y divisiones.
    \item Realizar operaciones lógicas y relacionales
    \item Poder generar funciones que reciban uno o más parámetros y que puedan regresar un resultado
    \item Una forma integrada en el lenguaje de programación para leer e imprimir resultados de la consola
    \item Manejo de variables locales y globales
    \item Manejo de arreglos integrados al lenguaje de programación
    \item Creación de números aleatorios en ciertas distribuciones integrados en el lenguaje de programación.
    \item Funciones integradas de las métricas centrales estadísticas (media,mediana,moda y varianza)
    \item Funciones de matemáticas integradas al sistema (exp,pow, sqrt entre otras)
    \item Manejo de vectores y matrices
\end{itemize}

Con estos requerimientos, también se ponderaron las limitaciones que iban a tener el lenguaje dado el corto tiempo y la falta de experiencia en este tipo de proyectos. Y se definieron las siguientes limitantes:

\begin{itemize}
    \item Solo se cuentan con un tiempo relativamente corto (Alrededor de 6 semanas)
    \item Es un proyecto individual, aunque se quiera expandir más el alcance del proyecto se tendrá que limitar y ajustar el trabajo a lo que es posible
    \item El lenguaje va a ser restrictivo para que sea posible hacerlo en el tiempo dado. Esto quiere decir que la implementación de múltiples scopes dentdo del lenguaje esta fuera del alcance del proyecto
\end{itemize}

Con estos requerimientos y limitantes en mente, se diseñaron las siguientes pruebas las cuales debería de ser posible hacerse en el lenguaje.

\begin{itemize}
    \item Se deberá poder programar un Hola Mundo en el lenguaje
    \item Se deberá poder programar una implementación secuencial y recursiva de la función factorial
    \item Se deberá poder programar una implementación secuencial y recursiva de la serie de Fibonacci
    \item Se deberá poder programar una implementación de un Bubble Sort y un Search en un arreglo
    \item Se deberá poder programar una multiplicación de matrices
    \item Se deberá poder hacer un programa que genere un arreglo que todos sus elementos pertenezcan a una distribución y que se le pueda hacer un análisis (aunque sea superficial) a los datos generados.
\end{itemize}

Una vez definido los requerimientos, las limitantes y las pruebas se empezó con el desarrollo del proyecto.


\section{Descripción del proceso general de desarrollo}

A lo largo del desarrollo del proyecto se estuvieron haciendo avances semanales en el compilador y la máquina virtual. Como medida de seguridad, se uso la herramienta de Git y Github para mantener un historial de control sobre los cambios hechos al proyecto. El repositorio de IntroProg puede ser encontrado en la liga https://github.com/LuisLomelinIbarra/Compiladores-2022. A pesar de mantener un control de versiones con Git, también se mantuvo una bitácora de desarrollo. A continuación se encuentra una bitácora de dichos avances en orden cronológico.

\subsubsection{Semana 1: Avance del Léxico Sintaxis}
Por el momento se realizó el lexer y parser con los diagramas en la propuesta. En este avance se realizó un programa de python que puede leer y aceptar (o rechazar) un archivo en el lenguaje de programación propuesto. Se realizaron los primeros dos archivos de prueba que son lenguaje valido. Estos son un helloworld y un archvio con operaciones y una función de factorial. En las pruebas realizadas regresa aceptado para los dos archivos. Faltan hacer más archivos de prueba para ver si se encuentran errores en el programa o si se debe de hacer un cambio a la gramática (en un caso extremo).

Actualmente el programa debe de:
\begin{itemize}
    \item Compilar sin errores
    \item Leer un archivo y decidir si es o no el lenguaje
    \item Indicar la linea del error si se encontró alguno
\end{itemize}

\subsection{Semana 2: Generación de la tabla de variables y el cubo semántico}
Se avanzo con la generación del directorio de funciones y la tabla de variables. Además, se creó el cubo semántico para las operaciones del lenguaje. De igual manera se empezó a implementar el análisis semántico de las operaciones del lenguaje. Al realizar esto se tuvo que reestructurar algunas reglas de gramática para acomodar mejor los puntos neurálgicos. También implementando el análisis se encontraron los siguientes problemas.
\begin{enumerate}
    \item Se podía generar arreglos de manera infinita ya que estaban al nivel de varcte. Como los arreglos textuales se pueden generar con expresiones, se podía anidar arreglos textuales en arreglos textuales. Para resolver esto se reestructuro la gramática para que solo se puedan utilizar en las asignaciones, las llamadas a función y las impresiones
    \item Hay que experimentar más con la verificación de los parámetros en una llamada a función. Esto hay que revisar que no rompa con el orden de las operaciones y probablemente hay que verlo mejor. En este avance no se pudo aplicar.
    \item No se sabe todavía cómo manejar las funciones de retorno
\end{enumerate}

Actualmente el programa debe de:
\begin{itemize}
    \item Detectar en las asignaciones que se haga la asignación correcta con respecto a el tipo de variable
    \item Detectar que se hagan las operaciones correctas con respecto a los tipos de los operandos en la mayoría de las operaciones
    \item Detectar que no haya definición múltiple de variables o arreglos
    \item Detectar que no haya definiciones múltiples de funciones
    \item Desplegar mensajes de error adecuados cuando haya errores de semántica
\end{itemize}



\subsection{Semana 3: Generar cuadruplos de expresiones}
En este avance ya se empezó a implementar los puntos neurálgicos de las expresiones. Actualmente, debe de poder generar el código intermedio de las expresiones en un archivo de texto llamado cuadruplos.txt. Falta probar bien todas las expresiones, pero con un archivo de pruebas rápidas con solo operaciones matemáticas parece ser que si genera bien el código. Se resolvieron algunos errores con la semántica para encontrar el tipo de un arreglo textual. También se cambió a que todos los elementos que se declaren en la función principal se declaren como su propia entrada en la entrada de principal en vez de que sea una adicción a la tabla global. También se arregló el error a la hora de declarar precedencia entre los operadores.

El avance de ahora debe de:
\begin{itemize}
    \item Generar código intermedio de expresiones en un archivo de cuadruplos
\end{itemize}


COSAS QUE INVESTIGAR:
\begin{enumerate}
    \item Hay que manejar bien la asignación de espacios en memoria de los cuadruplos para la generación de código intermedio. Por el momento solo va agregando de manera ascendente a un registro teoréticamente infinito
\end{enumerate}


\subsection{Semana 4: Código Intermedio para Condiciones y Bucles}
Se implemento la generación de cuádruplos para los ciclos y para las condiciones. Además, también se implementó el uso de direcciones virtuales en los cuádruplos generados. También se solucionaron ligeros errores que surgieron de algunos cambios menores a las expresiones. También ya se implementó la generación de los cuádruplos del estatuto imprimir (por fin). De igual manera se avanzaron con algunos puntos de módulos, principalmente agregar los parámetros correctos en el orden correcto a las funciones y la verificación de los tipos en las llamadas.

El avance de ahora debe de:
\begin{itemize}
    \item Generar código intermedio en un archivo de cuádruplos, generando también los cuádruplos pertinentes a los estatutos de condición y de bucle.
\end{itemize}


COSAS QUE INVESTIGAR:
\begin{enumerate}
    \item Como manejar los arreglos en memoria y como parámetros/argumentos de funciones.
\end{enumerate}



\subsection{Semana 5: Código Intermedio para Modulos (Funciones)}

Se implemento todo lo de módulos. Para cada función ahora se va guardando la cantidad de variables que se están utilizando en ella, el cuádruplo donde inicia y si tiene retorno la dirección de la variable global de retorno. En la llamada de función se verifica que los parámetros llamados sean los correctos, de ser así se genera el cuádruplo para generar la memoria y los cuádruplos para copiar los parámetros a la nueva memoria. Al final se agrega el gosubfunc que apunta al inicio de la función llamada. En caso de ser función especial llama a el cuádruplo SPFUNC, indicando que es función especial.

El avance de ahora debe de:

\begin{itemize}
    \item En el archivo de cuádruplos ahora ya debe de generar los cuádruplos correspondientes de cada función declarada (ERA y PARAMS). En caso de ser función especial llama al cuádruplo SPFUNC con el nombre de la función en vez de GOSUBFUNC.
\end{itemize}



COSAS QUE CAMBIAR/ARREGLAR:

\begin{enumerate}
    \item Una mejor manera para manejar las constantes String
    \item Corregir el error de la variable sintáctica que permite que se pueda declarar un arreglo así [1, [1,1]]
\end{enumerate}
   



\subsection{Semana 6: Máquina Virtual, Mapa de Memoria y ejecución de expresiones}

Ya se empezó la máquina virtual y ya funciona leyendo los cuádruplos de expresiones, el estatuto de imprimir y con las funciones definidas por el usuario (incluyendo recursión). Para este avance implemente una clase Memory que representa la memoria. Esta clase se puede utilizar para traducir las memorias virtuales asignadas en la compilación a memoria en la máquina virtual. Para lograr esto cada scope esta representado por un vector de dos instancias de la clase. Una de ellas representa las variables y la otra representa los temporales. En esta entrega me di cuenta de varios errores en la generación de cuádruplos, específicamente en el estatuto de return estaba faltando agregar el cuádruplo de enfunc para regresar a la memoria anterior. Igual mi lógica de saltos en mi for me llevaba a contar uno ciclo demás lo cual tuve que arreglar


El avance de ahora debe de:

\begin{itemize}
    \item El compilador debe de generar un archivo json con los cuádruplos, la tabla de funciones y las constantes.
    \item La máquina virtual debe de leer el json y ejecutar el código en él.
\end{itemize}

COSAS QUE CAMBIAR/ARREGLAR:

\begin{enumerate}
    \item Formas de manejar los arreglos como parametros
    \item Falta implementar la lógica de las funciones especiales
    \item Terminar bien arreglos.
\end{enumerate}


\subsection{Semana 7: Arreglos, Matrices y }

Ya está acabado el compilador. Actualmente el compilador recibe un archivo itp, con el lenguaje de IntroProg y genera un archivo obj (que es un json disfrazado) el cual es utilizado por la máquina virtual para ejecutar código. Es capaz de realizar el cálculo de factorial de manera secuencial y por función recursiva. También puede sacar el enésimo elemento de la secuencia Fibonacci de manera secuencial y de manera recursiva. También se puede hacer un Bubble Sort para organizar los elementos de un arreglo y se puede encontrar elementos dentro de los mismos.

También se le metió esfuerzo para que los mensajes de error sean lo más claro posibles.


El avance de ahora debe de:

\begin{itemize}
    \item El compilador debe de generar un archivo obj que puede ser leído por la máquina virtual.
    \item La máquina virtual debe de leer el obj y ejecutar el código en él, incluyendo funciones especiales de estadística.
\end{itemize}

COSAS QUE CAMBIAR/ARREGLAR:

\begin{enumerate}
    \item Actualizar los diagramas de sintaxis para reflejar los cambios que se hicieron durante la implementación
    \item Actualizar la gramática en la documentación 
    \item Acabar la documentación
\end{enumerate}

\subsection{Semana 8: Correciones y Empuje final de documentación}


Cambios mínimos al compilador. Se refactorizo lo más que se pudo sin romper lo existente y se agregaron comentarios pertinentes a las funciones. También se agregó un token extra para el manejo de comentarios en el lenguaje.


El avance de ahora debe de:

\begin{itemize}
    \item El compilador debe de generar un archivo obj que puede ser leído por la máquina virtual.
    \item La máquina virtual debe de leer el obj y ejecutar el código en él, incluyendo funciones especiales de estadística.
\end{itemize}

COSAS QUE CAMBIAR/ARREGLAR:

\begin{enumerate}
    \item Acabar la documentación
\end{enumerate}

\chapter{Descripción de IntroProg}

En esta sección se explica con más detalle lo que es el lenguaje y sus capacidades.
%---------------------------------------------------------------------------------------
\section{Introducción a IntroProg}\label{intro}

Es un lenguaje de programación sencillo con un enfoque hacia data science. La idea del lenguaje es permitir a personas interesadas en programar a experimentar con proyectos sencillos de ciencia de datos.

IntroProg permite al usuario aprender sobre la lógica de programación, permitiendo que el usuario pueda utilizar estructuras de decisión y estructuras de repetición. También permite que los usuarios experimenten un poco con funciones estadísticas, como generar números que pertenecen a distintas distribuciones de probabilidad como la función Normal, Exponencial, Poisson y Geométrica. Además, permite que los usuarios puedan crear gráficas de los datos con los que trabajan.


%---------------------------------------------------------------------------------------
\section{Características Principales}
IntroProg cuenta con múltiples características que la mayoría de los lenguajes de programación cuentan. Estos van de estructuras de control y flujo hasta el manejo de funciones y arreglos.
Como IntroProg es un lenguaje de aprendizaje puede ser un poco restrictivo, pero en esta sección se va a describir con un poco más de detalle en que consta cada una de las características del lenguaje, incluyendo las funciones especiales.

\FloatBarrier

\subsection{Estructura general de un programa de IntroProg}
Los archivos del lenguaje son archivos de texto con terminación \emph{itp}. Para empezar a escribir un programa en IntroProg primero se escribe la palabra clave \emph{programa} seguido por el nombre del programa. Después entre corchetas se escribe las declaraciones de variables, seguida por las declaraciones de funciones, seguido por la función principal. La función principal es la parte del código que se ejecuta primero.

\begin{figure}[!htbp]
    \centering
    
    \begin{lstlisting}
        programa pelos{
            // Primero las variables globales
            entero x;
            bool bo;
            // El codigo principal del programa
            principal funcion {
                // Variables locales a principal
                entero x,y;
            }{
                x = 10;
                y = 2;
                x = x*y;
                imprimir(x);
            }
        }
    \end{lstlisting}
    \caption{Ejemplo de la estructura de un programa de IntroProg}
\end{figure}
\FloatBarrier

Es importante mencionar que un programa puede o no tener una declaración de variables globales y declaración de funciones. Esto dependerá de las necesidades del programador cuando construya su código.

\begin{figure}[!htbp]
    \centering
    
    \begin{lstlisting}
        programa holaMundo { 
	    principal funcion {} {
		    imprimir("Hola Mundo!!!!");
	    } 
    }
    \end{lstlisting}
    \caption{Ejemplo de un programa sin declaración de variables ni declaración de funciones}
\end{figure}
\FloatBarrier

Aunque parezca un poco confuso al principio, se van a ir explicando parte por parte la estructura y funcionalidades del lenguaje en las siguientes secciones.

\subsection{Variables y la declaración de variables}

Para poder programar primero se necesitan utilizar variables. Las variables son representaciones de información que pueden ser modificadas por el programador para calcular y modelar información. Las variables en IntroProg están representadas por un identificador que es una serie de letras, guiones y números. Los identificadores siempre deben de empezar con una letra.

Para declarar una variable no solo es importante tener un buen identificador sino también hay que indicar el tipo de la variable. En IntroProg se manejan cuatro tipos de variables los cuales son:

\begin{itemize}
    \item Enteros: Números enteros (ej. 1,2,3).  Estos son representados por la palabra clave \emph{entero}
    \item Flotantes: Números Flotantes o decimales (ej. 3.1416), representados por la palabra clave \emph{flotante} 
    \item Caracteres: Letras (ej. 'a', 'b'). Estas estan representadas por la palabra clave \emph{char}
    \item Booleanos: Valores Booleanos (ej. Verdadero o falso). Esta representados por la palabra clave \emph{bool}
\end{itemize}

\begin{figure}[!htbp]
    \centering
    
    \begin{lstlisting}
        entero una_Variable_1;
    \end{lstlisting}
    \caption{Ejemplo de una declaración de variable}
\end{figure}
\FloatBarrier
Las declaraciones solo se pueden hacer en los lugares apropiados en el programa para las declaraciones. Estos son el espacio de variables globales que se encuentra directamente después de la primera corcheta del programa, y en las corchetas de declaración de variables locales los cuales se encuentran después de los nombres de funciones.

Los programas cuentan con dos tipos de alcances de variables. Las variables globales, las cuales pueden ser accesados por todas las funciones del programa y las variables locales que solo pueden ser accesados por las funciones en las que fueron declaradas.

\subsection{Expresiones, Imprimir, Asignación y Comentarios}

Solo declarar las variables no es suficiente para hacer programas. Por ello IntroProg cuenta con una serie de estatutos básicos que son utilizados para realizar operaciones con las variables declaradas. Primero para poder asignarles información a las variables se tiene que hacer una asignación. Esto se logra utilizando el símbolo de \emph{\=}. Para asignar valores a variables primero se tiene que escribir el identificador de la variable seguido por el símbolo \emph{\=} seguido del valor que se le quiere asignar.

\begin{figure}[!htbp]
    \centering
    
    \begin{lstlisting}
        una_Variable_1 = 10;
        // Tambien se le pueden asignar 
        // el valor de otra variable
        una_Variable_1 = x; 
    \end{lstlisting}
    \caption{Ejemplo de asignación}
\end{figure}
\FloatBarrier
Hay que tomar en consideración que solo se le pueden asignar valores a las variables que sean del tipo de la variable. En caso de que se le intente asignar un valor a la variable que no sea de su tipo, IntroProg va a intentar convertir el valor que va a ser asignado al tipo de la variable. Si no puede hacerlo el compilador va a levantar un error.

Ya con valores definidos se pueden empezar a hacer expresiones. Las expresiones en IntroProg son como las expresiones de matemáticas. Estas hacen una operación sobre las variables y constantes y generan un resultado. Estos resultados pueden ser guardados en otras variables.

\begin{figure}[!htbp]
    \centering
    
    \begin{lstlisting}
        una_Variable_1 = 10 * una_Variable_1 + 1;

    \end{lstlisting}
    \caption{Ejemplo de asignación con expresiones}
\end{figure}
\FloatBarrier

IntroProg sigue la siguiente jerquía de operaciones:
\begin{itemize}
    \item Parentesis : Simpre va a hacer primero lo que hay entre parentesis
    \item Negativos/Positivos: Primero se hace negativo o positivo el valor de la variable o constante (ej. -1, +x)
    \item Multiplicaciones/Divisiones : Operaciones de multiplicación y División
    \item Sumas/Restas : Operaciones de sumas y restas
    \item Relacionales : Operaciones relacionales, las cuales son mayor (\emph{>}), menor (\emph{<}), igual (\emph{==}), mayor o igual (\emph{>=}), menor o igual (\emph{<=}) y diferente (\emph{!=})
    \item Lógicas : Operaciones Lógicas como \emph{o (||)} e \emph{y (\&\&)}
\end{itemize}

Es importante mencionar que todos los estatutos o expresiones que se hagan en IntroProg deben de terminar en un punto y coma. De otra manera el compilador va a marcar error.

Pero no solo es necesario hacer operaciones, también es importante saber el resultado de las operaciones. Para poder ver estos resultados es importante poder imprimir a pantalla los valores de las variables. Por eso IntroProg provee el estatuto de \emph{imprimir} el cual permite imprimir a consola el valor de alguna variable o expresión. Para usar este estatuto se tiene que usar la palabra clave \emph{imprimir} seguida por todos los elementos a imprimir separados por comas.

\begin{figure}[!htbp]
    \centering
    
    \begin{lstlisting}
        una_Variable_1 = 10 * una_Variable_1 + 1;
        imprimir(una_Variable,10);

    \end{lstlisting}
    \caption{Ejemplo de impresión}
\end{figure}
\FloatBarrier



Este estatuto también puede manejar el uso de constantes de cadenas. Las constantes de cadenas son oraciones o palabras que estan entre comillas. Esto puede ser util para organizar bien la información cuando se despliega en pantalla.

\begin{figure}[!htbp]
    \centering
    
    \begin{lstlisting}
        una_Variable_1 = 10 * una_Variable_1 + 1;
        imprimir("Esto es una cadena : ",una_Variable,10);

    \end{lstlisting}
    \caption{Ejemplo del uso de cadenas en impresión}
\end{figure}
\FloatBarrier


Finalmente IntroProg cuenta con un estatuto especial que puede ayudar a los programadores a organizar su códgio. Este estatuto es el comentario. El comentario se escribe como dos diagonales seguido de lo que el programador quiera escribir. Todo lo que siga después de las diagonales va a ser ignorado por el compilador.

\begin{figure}[!htbp]
    \centering
    
    \begin{lstlisting}
        una_Variable_1 = 10 * una_Variable_1 + 1;
        imprimir(una_Variable,10); // el comentario puede ser lo que sea
        // El compilador ignora esta linea
        // Y no casua errores

    \end{lstlisting}
    \caption{Ejemplo de Comentarios}
\end{figure}
\FloatBarrier

\subsection{Condiciones}
Las estructuras de control más comúnmente utilizadas. En la mayoría de los lenguajes son reconocido como la instrucción o palabra clave \emph{if}. En IntroProg esta estructura está presente bajo el nombre de \emph{si}, y como su contraparte en C/C++ funciona de manera similar.
\subsubsection{Estructura si}
La estructura \emph{si} está compuesta por la palabra clave seguida de una expresión entre paréntesis. Después de este paréntesis se escribe un bloque de código entre corchetas. El comportamiento de si es el siguiente.
 \emph{Si} recibe una expresión booleana entre paréntesis y si la expresión resulta en verdadero ejecuta el código que se encuentra en los corchetes próximos a la estructura. Si la expresión es falsa se salta a la primera línea de código que sigue después de las corchetas.

\begin{figure}[!htbp]
    \centering
    
    \begin{lstlisting}
        si(expresion){
            //Codigo a ejecutar en verdadero
        }
        //Continua codigo
    \end{lstlisting}
    \caption{Ejemplo de \emph{si}}
\end{figure}
\FloatBarrier
\subsubsection{Estructura si-sino}
Similar a sus contrapartes IntroProg también cuenta con la estructura de \emph{if-else}. En IntroProg esta estructura se escribe con las palabras clave \emph{si} y \emph{sino}. Su funcionamiento es el siguiente. Como un \emph{si} regular, primero se evalúa la expresión que se encuentra entre paréntesis enfrente de la palabra clave, si es verdadero ejecuta el código entre las corchetas seguidas de la expresión. En caso de que la condición sea falsa se salta a ejecutar las corchetas después de la palabra clave \emph{sino}.

\begin{figure}[!htbp]
    \centering
    
    \begin{lstlisting}
        si(expresion){
            //Codigo a ejecutar en verdadero
        }sino{
            //Codigo a ejecutar en falso
        }
        //Continua codigo
    \end{lstlisting}
    \caption{Ejemplo de \emph{si-sino}}
\end{figure}
\FloatBarrier

\FloatBarrier
\subsection{Bucles}

Los bucles son una de las estructuras de flujo esenciales de los lenguajes de programación. Y como todos los lenguajes, IntroProg también cuenta con formas de representar estas estructuras. En IntroProg hay dos formas de hacer bucles. La primera estructura es \emph{mientras}, el cual es muy similar a su contraparte de C/C++ \emph{while}. La segunda estructura es \emph{por} el cual es similar a su contraparte de C/C++ \emph{for}.
\\
\subsubsection{Estructura Mientras}
La estructura de \emph{mientras} consta de la palabra clave seguida de una expresión entre paréntesis y un bloque de código entre corchetas. Lo que va a hacer es repetir el bloque de código mientras que la expresión que se encuentra entre paréntesis enfrente del \emph{mientras}. Si la expresión resulta en falso se salta a la siguiente línea de código después de la estructura.

\begin{figure}[!htbp]
    \centering
    
    \begin{lstlisting}
        mientras(expresion){
            //Codigo a repetir mientras sea verdadero
        }
        // Codigo continua aqui
    \end{lstlisting}
    \caption{Ejemplo de \emph{mientras}}
\end{figure}
\FloatBarrier

\subsubsection{Estructura Por}
La segunda estructura de flujo de IntroProg es la estructura \emph{por}, la cual es similar a su contraparte \emph{for} en el lenguaje C/C++. La estructura consta de la palabra \emph{por} seguida por una asignación, una expresión a evaluar y una segunda asignación indicando el paso a seguir después de cada ciclo encerrado entre un paréntesis. Seguido de este paréntesis sigue un bloque de código entre corchetas el cual va a ser ejecutado mientras que la expresión del \emph{por} sea verdadera.

\begin{figure}[!htbp]
    \centering
    
    \begin{lstlisting}
        por(asignacion; expresion; asignacion;){
            //Codigo a repetir
        }
        //Codigo continua
    \end{lstlisting}
    \caption{Ejemplo de \emph{por}}
\end{figure}
\FloatBarrier

El \emph{por} también se puede escribir sin la primera asignación, asumiendo que la variables utilizadas en el paso y la expresión estén declaradas.


\begin{figure}[!htbp]
    \centering
    
    \begin{lstlisting}
        por(; expresion; asignacion;){
            //Codigo a repetir
        }
        //Codigo continua
    \end{lstlisting}
    \caption{Ejemplo de \emph{por} sin la primera asignación}
\end{figure}

\FloatBarrier
\subsection{Funciones}

Otra parte importante de los lenguajes de programación es la posibilidad de poder romper el código en funciones que cumplen un funcionamiento especifico, definido por el programador. IntroProg estructura sus funciones de la siguiente manera.

\subsubsection{Declaración de función}

Antes de utilizar funciones es importante primero que el usuario declare las funciones. Para ello hay que escribir la declaración de las funciones en el espacio apropiado. En la estructura de un programa de IntroProg las funciones se declaran después de las declaraciones de las variables globales, pero antes de la función principal.

\begin{figure}[!htbp]
    \centering
    
    \begin{lstlisting}
        programa pelos{
            // Primero las variables globales
            entero x;
            bool bo;
            // Aqui van las funciones
            funcion vacio holaMundo(){}{
                imprimir("Hola Mundo!!!");
            }
            // El codigo principal del programa
            principal funcion {
                // Variables locales a principal
                entero x,y;
            }{
                x = 10;
                y = 2;
                x = x*y;
                imprimir(x);
                $holaMundo();
            }
        }
    \end{lstlisting}
    \caption{Ejemplo del lugar donde se declaran las funciones}
\end{figure}
\FloatBarrier

El programador puede declarar una función usando la palabra clave \emph{funcion} seguida por el tipo de función. Después se escribe los parámetros de la función, escritos entre paréntesis denotando el tipo y el identificador del parámetro. Es importante remarcar que una función puede no tener parámetros. Se pueden definir múltiples parámetros separando con comas cada uno de ellos. Después de definir los parámetros se escribe entre corchetas las declaraciones de variables. Si no se planea usar variables locales en la función se puede dejar vació el espacio entre los corchetes. Después de declarar las variables se escribe el bloque de código que debe de ejecutar la función entre corchetas.

\begin{figure}[!htbp]
    \centering
    
    \begin{lstlisting}
        funcion vacio holaMundo(){}{
            imprimir("Hola Mundo!!!");
        }
    \end{lstlisting}
    \caption{Ejemplo de función vacia}
\end{figure}
\FloatBarrier

\begin{figure}[!htbp]
    \centering
    
    \begin{lstlisting}
        funcion entero suma( entero n, entero m){}{
            regresa n * m;
        }
    \end{lstlisting}
    \caption{Ejemplo de función con retorno}
\end{figure}
\FloatBarrier

En IntroProg hay dos tipos de funciones, las funciones vacías y las funciones con retorno. Las funciones vacías sirven solo para ejecutar código que no se esperan que regresen un valor, esto puede ser imprimir el resultado de ciertas variables o hacer cálculos temporales que no se van a utilizar después. Las funciones con retorno son funciones que realizan operaciones y que regresan un resultado. El valor de retorno de estas funciones están indicadas por la palabra clave \emph{regresar} seguida por una expresión el cual su resultado es el que va a ser utilizado. Cabe recalcar que las funciones vacío no pueden tener el estatuto de \emph{regresar}.


\subsubsection{Llamada de función}

Una vez declarada las funciones pueden ser utilizadas como el elemento de una expresión. Esto se logra utilizando el símbolo de dólar (\emph{\$}) seguido por el nombre de la función, seguido por un paréntesis con expresiones separados por comas, representando los argumentos. Si la función no tiene argumentos se dejan los paréntesis vacíos.

\begin{figure}[!htbp]
    \centering
   
    \begin{lstlisting}
            principal funcion {
                // Variables locales a principal
                entero x,y;
            }{
                x = 10;
                y = 2;
                x = $suma(x, x*y);
                imprimir(x);
                $holaMundo();
            }
    \end{lstlisting}
     \caption{Ejemplo del lugar donde se declaran las funciones}
\end{figure}
\FloatBarrier
\subsubsection{Llamadas recursivas en funciones}

IntroProg también soporta el uso de funciones recursivas. Esto se puede lograr haciendo la llamada de la misma función dentro de su declaración. Esto puede ser útil para resolver ciertos problemas de programación.

\begin{figure}[!htbp]
    \centering
    
    \begin{lstlisting}
        funcion entera factorial ( entero  n) {}{
            regresa n * $factorial(n-1);
        } 
    \end{lstlisting}
    \caption{Ejemplo de función recursiva}
\end{figure}


\FloatBarrier
\subsection{Arreglos, Matrices y Cubos}

En IntroProg las variables no solo se pueden usar para guardar solo un valor. Se pueden declarar arreglos los cuales pueden tener múltiples valores dependiendo de su declaración. Para declarar un arreglo se usa una sintaxis muy similar a la declaración simple. La declaración de un arreglo se da escribiendo primero el tipo de la variable, seguido por su identificador, seguido del tamaño del arreglo encerrado entre corchetas cuadradas.\\

\begin{figure}[!htbp]
    \centering
    
    \begin{lstlisting}
        entero arreglo [10]; // Un arreglo de 10 valores

    \end{lstlisting}
    \caption{Ejemplo de la declaración de un arreglo}
\end{figure}
\FloatBarrier


Para poder accesar a uno de los valores que tiene el arreglo se puede llamar en una expresión de la siguiente manera. Se escribe el nombre del arreglo seguido de la posición del valor entre corchetas cuadradas. Las posiciones en IntroProg se empiezan a contar desde el 0 hasta el número entero anterior al tamaño.\\

\begin{figure}[!htbp]
    \centering
    
    \begin{lstlisting}
        arreglo[0] = 10; // primera posicion del arreglo
        arreglo[9] = 2003; // Ultima posicion del arreglo
        //Tambien se pueden usar para operaciones
        arreglo[2] = arreglo[0] * arreglo[9];

    \end{lstlisting}
    \caption{Ejemplo de la llamada de un arreglo}
\end{figure}
\FloatBarrier
Los arreglos en IntroProg también se les pueden asignar valores de manera directa a todas las casillas del arreglo. Esto se puede lograr listando los valores entre corchetas cuadradas y separando los valores entre comas.

\begin{figure}[!htbp]
    \centering
    
    \begin{lstlisting}
        arreglo = [1,2,3,4,5,6,7,8,9,10]

    \end{lstlisting}
    \caption{Ejemplo de una asignación textual a un arreglo}
\end{figure}
\FloatBarrier
\subsubsection{Matrices y Cubos}
IntroProg no solo maneja arreglos sino también maneja matrices y cubos. La forma de interactuar con ellos es muy similar a los arreglos. Solo se necesita hacer lo mismo que se hace con un arreglo solo que en vez de utilizar solo un set de corchetas cuadradas se usan la cantidad de corchetas cuadradas a las dimensiones a utilizar. Es importante mencionar que IntroProg solo maneja hasta cubos y no permite crear variables con más dimensiones.

\begin{figure}[!htbp]
    \centering
    
    \begin{lstlisting}
        entero mat[3][3];
        entero cubo[2][2][2];

    \end{lstlisting}
    \caption{Ejemplo de la declaración de una matriz y un cubo}
\end{figure}
\FloatBarrier

\begin{figure}[!htbp]
    \centering
    
    \begin{lstlisting}
        mat[0][0] = 10;
        cubo[0][1][1] = 10;

    \end{lstlisting}
    \caption{Ejemplo de la llamada de una matriz y un cubo}
\end{figure}
\FloatBarrier

\begin{figure}[!htbp]
    \centering
    
    \begin{lstlisting}
        mat = [[1,2,3],[4,5,6],[7,8,9]];
        cubo = [[[1,2],[3,4]],[[5,6],[7,8]]];

    \end{lstlisting}
    \caption{Ejemplo de una asignación textual de una matriz y un cubo}
\end{figure}
\FloatBarrier






\FloatBarrier
\subsection{Funciones Especiales}

IntroProg también cuenta con una serie de funciones especiales que pueden ser utilizadas en cualquier programa. Estas funciones tienen un enfoque hacia estadística, y le permiten a los usuarios hacer programas más complejos. Estas funciones son:

\begin{itemize}
    \item \$leer() : Regresa la lectura capturada de consola
    \item \$modulo(flotante a, flotante b): Regresa el resultado de a\%b
    \item \$suma(flotante a[]): Suma todos los elementos de un arreglo y regresa sus resultados
    \item \$raiz(flotante a): Regresa el flotante resultante de la raíz de a
    \item \$exp(flotante a): Regresa la exponencial de e\^a
    \item \$elevar(flotante a, flotante b): Regresa el resultado de a\^b
    \item \$techo(flotante a): Regresa un flotante a redondeado para arriba
    \item \$piso(flotante a): Regresa un flotante a redondeado para abajo
    \item \$cos(flotante a): Regresa el coseno de a
    \item \$sen(flotante a): Regresa el seno de a
    \item \$tan(flotante a): Regresa la tangente de a
    \item \$cotan(flotante a): Regresa la cotangente de a
    \item \$sec(flotante a): Regresa la secante de a
    \item \$cosec(flotante a): Regresa la cosecante de a
    \item \$log(flotante a): Regresa el logaritmo natural de a
    \item \$minimo(flotante a[]): Regresa el valor más chico en el vector a
    \item \$maximo(flotante a[]): Regresa el valor máximo de a
    \item \$redondear(flotante a): Regresa un flotante a redondeado 
    \item \$productoPunto(flotante a[], flotante b[]): Regresa el producto punto entre los vectores de entrada a y b
    \item \$media(flotante a[]): Regresa la media de a
    \item \$mediana(flotante a[]): regresa la mediana de a
    \item \$moda(flotante a[]): Regresa el elemento con la moda más alta de a
    \item \$varianza(a): Regresa la varianza de a
    \item \$percentil(flotante a[], flotante q): Regresa el valor en el que se encuentran q% de los valores en a
    \item \$aleatorio(flotante min, flotante max): Regresa un número flotante aleatorio entre los rangos de argumentos mínimos y máximos
    \item \$wilcoxon(flotante x[]): Realiza la prueba de Wilcoxon en la serie de datos en x
    \item \$wilcoxonComp(flotante x[], flotante y[]):Realiza la prueba de Wilcoxon se realiza la prueba sobre los datos x y y.
    \item \$regresionSimple(flotante x[], flotante y[], flotante xi): Dado un set de x y y se usará regresión lineal simple para encontrar f(xi) y se regresara ese valor
    \item \$normal(flotante media, flotante desv): Regresa un número escalar que pertenezca a la distribución normal dado los parámetros
    \item \$poisson(flotante lambda): Regresa un número aleatorio de la distribución Poisson con la lambda dada
    \item \$dexponencial(flotante beta): Regresa un número aleatorio de la distribución Exponencial correspondiente a la beta (o 1/Lambda) dada.
    \item \$dgeometrica(flotante exito): Te regresa un valor con la distribución geométrica con la probabilidad de exito dada. 
    \item \$histograma(flotante x[], flotante rango): Genera un histograma a partir de los datos en el vector de x con un rango entre los datos de rango.
    \item \$diagramadecaja(flotante x[]): Genera un diagrama de caja y bigotes de los datos en en x
    \item \$grafDispersion(flotante x[], flotante y[]): Genera un gráfico de dispersión con los valores de ‘x’ y ‘y’
\end{itemize}

Para llamar a una función especial se usa la misma sintaxis que una función normal.

\begin{figure}[!htbp]
    \centering
    
    \begin{lstlisting}
            x = $aleatorio(1,100);
    \end{lstlisting}
    \caption{Ejemplo del uso de funcinoes especiales}
\end{figure}
\FloatBarrier

\FloatBarrier

%---------------------------------------------------------------------------------------

\section{Listado de errores}
El compilador y la máquina virtual manejan distintos tipos de errores que pueden suceder cuando se hace un programa. Como el lenguaje está diseñado para ser un punto de entrada a la programación se intentó cubrir la mayor cantidad de errores de programación comunes y se agregaron mensajes de error apropiados a la situación del error. Estos mensajes de error tratan de explicar de una manera clara cuál es la situación que causo el error al usuario.\\

Los errores que se pueden detectar durante compilación son los siguientes:
\begin{itemize}
    \item Doble declaración de variables
    \item El uso de variables no declaradas.
    \item Errores de sintaxis. Esto incluye identificadores y palabras reservadas mal escritas, estatutos mal escritos entre otros errores de sintaxis.
    \item Número de parámetros incorrectos
    \item Asignación incorrecta de parámetros, si el valor pasado como parámetro no se puede convertir a al tipo del parámetro el compilador levanta un error.
    \item Operaciones con funciones vacías, el compilador no permite el uso de funciones vacías en expresiones
    \item Retornos en funciones vacías, El compilador levanta errores cuando detecta un retorno cuando hay retornos en funciones vacías
    \item No permitir el nombre de funciones y palabras reservadas como identificadores.
\end{itemize}

Los errores que puede detectar la máquina virtual son la siguiente:
\begin{itemize}
    \item Operaciones con variables sin valor. Si se intenta realizar operaciones con variables las cuales no cuentan con un valor se levanta un error y se le indica al usuario cuál fue el posible error.
    \item Detectar si los subíndices del arreglo se salen del espacio definido por el arreglo. Similarmente detecta cuando el arreglo se sale de los límites del arreglo e indica al usuario mediante un mensaje de error lo que sucedió.
    \item Lectura incorrecta. Cuando se utiliza la lectura y se mete un valor que no es válido para lectura.
\end{itemize}




%\part{Descripción del lenguaje}
%\chapter{Descripción de IntroProg}

En esta sección se explica con más detalle lo que es el lenguaje y sus capacidades.
%---------------------------------------------------------------------------------------
\section{Introducción a IntroProg}\label{intro}

Es un lenguaje de programación sencillo con un enfoque hacia data science. La idea del lenguaje es permitir a personas interesadas en programar a experimentar con proyectos sencillos de ciencia de datos.

IntroProg permite al usuario aprender sobre la lógica de programación, permitiendo que el usuario pueda utilizar estructuras de decisión y estructuras de repetición. También permite que los usuarios experimenten un poco con funciones estadísticas, como generar números que pertenecen a distintas distribuciones de probabilidad como la función Normal, Exponencial, Poisson y Geométrica. Además, permite que los usuarios puedan crear gráficas de los datos con los que trabajan.


%---------------------------------------------------------------------------------------
\section{Características Principales}
IntroProg cuenta con múltiples características que la mayoría de los lenguajes de programación cuentan. Estos van de estructuras de control y flujo hasta el manejo de funciones y arreglos.
Como IntroProg es un lenguaje de aprendizaje puede ser un poco restrictivo, pero en esta sección se va a describir con un poco más de detalle en que consta cada una de las características del lenguaje, incluyendo las funciones especiales.

\FloatBarrier

\subsection{Estructura general de un programa de IntroProg}
Los archivos del lenguaje son archivos de texto con terminación \emph{itp}. Para empezar a escribir un programa en IntroProg primero se escribe la palabra clave \emph{programa} seguido por el nombre del programa. Después entre corchetas se escribe las declaraciones de variables, seguida por las declaraciones de funciones, seguido por la función principal. La función principal es la parte del código que se ejecuta primero.

\begin{figure}[!htbp]
    \centering
    
    \begin{lstlisting}
        programa pelos{
            // Primero las variables globales
            entero x;
            bool bo;
            // El codigo principal del programa
            principal funcion {
                // Variables locales a principal
                entero x,y;
            }{
                x = 10;
                y = 2;
                x = x*y;
                imprimir(x);
            }
        }
    \end{lstlisting}
    \caption{Ejemplo de la estructura de un programa de IntroProg}
\end{figure}
\FloatBarrier

Es importante mencionar que un programa puede o no tener una declaración de variables globales y declaración de funciones. Esto dependerá de las necesidades del programador cuando construya su código.

\begin{figure}[!htbp]
    \centering
    
    \begin{lstlisting}
        programa holaMundo { 
	    principal funcion {} {
		    imprimir("Hola Mundo!!!!");
	    } 
    }
    \end{lstlisting}
    \caption{Ejemplo de un programa sin declaración de variables ni declaración de funciones}
\end{figure}
\FloatBarrier

Aunque parezca un poco confuso al principio, se van a ir explicando parte por parte la estructura y funcionalidades del lenguaje en las siguientes secciones.

\subsection{Variables y la declaración de variables}

Para poder programar primero se necesitan utilizar variables. Las variables son representaciones de información que pueden ser modificadas por el programador para calcular y modelar información. Las variables en IntroProg están representadas por un identificador que es una serie de letras, guiones y números. Los identificadores siempre deben de empezar con una letra.

Para declarar una variable no solo es importante tener un buen identificador sino también hay que indicar el tipo de la variable. En IntroProg se manejan cuatro tipos de variables los cuales son:

\begin{itemize}
    \item Enteros: Números enteros (ej. 1,2,3).  Estos son representados por la palabra clave \emph{entero}
    \item Flotantes: Números Flotantes o decimales (ej. 3.1416), representados por la palabra clave \emph{flotante} 
    \item Caracteres: Letras (ej. 'a', 'b'). Estas estan representadas por la palabra clave \emph{char}
    \item Booleanos: Valores Booleanos (ej. Verdadero o falso). Esta representados por la palabra clave \emph{bool}
\end{itemize}

\begin{figure}[!htbp]
    \centering
    
    \begin{lstlisting}
        entero una_Variable_1;
    \end{lstlisting}
    \caption{Ejemplo de una declaración de variable}
\end{figure}
\FloatBarrier
Las declaraciones solo se pueden hacer en los lugares apropiados en el programa para las declaraciones. Estos son el espacio de variables globales que se encuentra directamente después de la primera corcheta del programa, y en las corchetas de declaración de variables locales los cuales se encuentran después de los nombres de funciones.

Los programas cuentan con dos tipos de alcances de variables. Las variables globales, las cuales pueden ser accesados por todas las funciones del programa y las variables locales que solo pueden ser accesados por las funciones en las que fueron declaradas.

\subsection{Expresiones, Imprimir, Asignación y Comentarios}

Solo declarar las variables no es suficiente para hacer programas. Por ello IntroProg cuenta con una serie de estatutos básicos que son utilizados para realizar operaciones con las variables declaradas. Primero para poder asignarles información a las variables se tiene que hacer una asignación. Esto se logra utilizando el símbolo de \emph{\=}. Para asignar valores a variables primero se tiene que escribir el identificador de la variable seguido por el símbolo \emph{\=} seguido del valor que se le quiere asignar.

\begin{figure}[!htbp]
    \centering
    
    \begin{lstlisting}
        una_Variable_1 = 10;
        // Tambien se le pueden asignar 
        // el valor de otra variable
        una_Variable_1 = x; 
    \end{lstlisting}
    \caption{Ejemplo de asignación}
\end{figure}
\FloatBarrier
Hay que tomar en consideración que solo se le pueden asignar valores a las variables que sean del tipo de la variable. En caso de que se le intente asignar un valor a la variable que no sea de su tipo, IntroProg va a intentar convertir el valor que va a ser asignado al tipo de la variable. Si no puede hacerlo el compilador va a levantar un error.

Ya con valores definidos se pueden empezar a hacer expresiones. Las expresiones en IntroProg son como las expresiones de matemáticas. Estas hacen una operación sobre las variables y constantes y generan un resultado. Estos resultados pueden ser guardados en otras variables.

\begin{figure}[!htbp]
    \centering
    
    \begin{lstlisting}
        una_Variable_1 = 10 * una_Variable_1 + 1;

    \end{lstlisting}
    \caption{Ejemplo de asignación con expresiones}
\end{figure}
\FloatBarrier

IntroProg sigue la siguiente jerquía de operaciones:
\begin{itemize}
    \item Parentesis : Simpre va a hacer primero lo que hay entre parentesis
    \item Negativos/Positivos: Primero se hace negativo o positivo el valor de la variable o constante (ej. -1, +x)
    \item Multiplicaciones/Divisiones : Operaciones de multiplicación y División
    \item Sumas/Restas : Operaciones de sumas y restas
    \item Relacionales : Operaciones relacionales, las cuales son mayor (\emph{>}), menor (\emph{<}), igual (\emph{==}), mayor o igual (\emph{>=}), menor o igual (\emph{<=}) y diferente (\emph{!=})
    \item Lógicas : Operaciones Lógicas como \emph{o (||)} e \emph{y (\&\&)}
\end{itemize}

Es importante mencionar que todos los estatutos o expresiones que se hagan en IntroProg deben de terminar en un punto y coma. De otra manera el compilador va a marcar error.

Pero no solo es necesario hacer operaciones, también es importante saber el resultado de las operaciones. Para poder ver estos resultados es importante poder imprimir a pantalla los valores de las variables. Por eso IntroProg provee el estatuto de \emph{imprimir} el cual permite imprimir a consola el valor de alguna variable o expresión. Para usar este estatuto se tiene que usar la palabra clave \emph{imprimir} seguida por todos los elementos a imprimir separados por comas.

\begin{figure}[!htbp]
    \centering
    
    \begin{lstlisting}
        una_Variable_1 = 10 * una_Variable_1 + 1;
        imprimir(una_Variable,10);

    \end{lstlisting}
    \caption{Ejemplo de impresión}
\end{figure}
\FloatBarrier



Este estatuto también puede manejar el uso de constantes de cadenas. Las constantes de cadenas son oraciones o palabras que estan entre comillas. Esto puede ser util para organizar bien la información cuando se despliega en pantalla.

\begin{figure}[!htbp]
    \centering
    
    \begin{lstlisting}
        una_Variable_1 = 10 * una_Variable_1 + 1;
        imprimir("Esto es una cadena : ",una_Variable,10);

    \end{lstlisting}
    \caption{Ejemplo del uso de cadenas en impresión}
\end{figure}
\FloatBarrier


Finalmente IntroProg cuenta con un estatuto especial que puede ayudar a los programadores a organizar su códgio. Este estatuto es el comentario. El comentario se escribe como dos diagonales seguido de lo que el programador quiera escribir. Todo lo que siga después de las diagonales va a ser ignorado por el compilador.

\begin{figure}[!htbp]
    \centering
    
    \begin{lstlisting}
        una_Variable_1 = 10 * una_Variable_1 + 1;
        imprimir(una_Variable,10); // el comentario puede ser lo que sea
        // El compilador ignora esta linea
        // Y no casua errores

    \end{lstlisting}
    \caption{Ejemplo de Comentarios}
\end{figure}
\FloatBarrier

\subsection{Condiciones}
Las estructuras de control más comúnmente utilizadas. En la mayoría de los lenguajes son reconocido como la instrucción o palabra clave \emph{if}. En IntroProg esta estructura está presente bajo el nombre de \emph{si}, y como su contraparte en C/C++ funciona de manera similar.
\subsubsection{Estructura si}
La estructura \emph{si} está compuesta por la palabra clave seguida de una expresión entre paréntesis. Después de este paréntesis se escribe un bloque de código entre corchetas. El comportamiento de si es el siguiente.
 \emph{Si} recibe una expresión booleana entre paréntesis y si la expresión resulta en verdadero ejecuta el código que se encuentra en los corchetes próximos a la estructura. Si la expresión es falsa se salta a la primera línea de código que sigue después de las corchetas.

\begin{figure}[!htbp]
    \centering
    
    \begin{lstlisting}
        si(expresion){
            //Codigo a ejecutar en verdadero
        }
        //Continua codigo
    \end{lstlisting}
    \caption{Ejemplo de \emph{si}}
\end{figure}
\FloatBarrier
\subsubsection{Estructura si-sino}
Similar a sus contrapartes IntroProg también cuenta con la estructura de \emph{if-else}. En IntroProg esta estructura se escribe con las palabras clave \emph{si} y \emph{sino}. Su funcionamiento es el siguiente. Como un \emph{si} regular, primero se evalúa la expresión que se encuentra entre paréntesis enfrente de la palabra clave, si es verdadero ejecuta el código entre las corchetas seguidas de la expresión. En caso de que la condición sea falsa se salta a ejecutar las corchetas después de la palabra clave \emph{sino}.

\begin{figure}[!htbp]
    \centering
    
    \begin{lstlisting}
        si(expresion){
            //Codigo a ejecutar en verdadero
        }sino{
            //Codigo a ejecutar en falso
        }
        //Continua codigo
    \end{lstlisting}
    \caption{Ejemplo de \emph{si-sino}}
\end{figure}
\FloatBarrier

\FloatBarrier
\subsection{Bucles}

Los bucles son una de las estructuras de flujo esenciales de los lenguajes de programación. Y como todos los lenguajes, IntroProg también cuenta con formas de representar estas estructuras. En IntroProg hay dos formas de hacer bucles. La primera estructura es \emph{mientras}, el cual es muy similar a su contraparte de C/C++ \emph{while}. La segunda estructura es \emph{por} el cual es similar a su contraparte de C/C++ \emph{for}.
\\
\subsubsection{Estructura Mientras}
La estructura de \emph{mientras} consta de la palabra clave seguida de una expresión entre paréntesis y un bloque de código entre corchetas. Lo que va a hacer es repetir el bloque de código mientras que la expresión que se encuentra entre paréntesis enfrente del \emph{mientras}. Si la expresión resulta en falso se salta a la siguiente línea de código después de la estructura.

\begin{figure}[!htbp]
    \centering
    
    \begin{lstlisting}
        mientras(expresion){
            //Codigo a repetir mientras sea verdadero
        }
        // Codigo continua aqui
    \end{lstlisting}
    \caption{Ejemplo de \emph{mientras}}
\end{figure}
\FloatBarrier

\subsubsection{Estructura Por}
La segunda estructura de flujo de IntroProg es la estructura \emph{por}, la cual es similar a su contraparte \emph{for} en el lenguaje C/C++. La estructura consta de la palabra \emph{por} seguida por una asignación, una expresión a evaluar y una segunda asignación indicando el paso a seguir después de cada ciclo encerrado entre un paréntesis. Seguido de este paréntesis sigue un bloque de código entre corchetas el cual va a ser ejecutado mientras que la expresión del \emph{por} sea verdadera.

\begin{figure}[!htbp]
    \centering
    
    \begin{lstlisting}
        por(asignacion; expresion; asignacion;){
            //Codigo a repetir
        }
        //Codigo continua
    \end{lstlisting}
    \caption{Ejemplo de \emph{por}}
\end{figure}
\FloatBarrier

El \emph{por} también se puede escribir sin la primera asignación, asumiendo que la variables utilizadas en el paso y la expresión estén declaradas.


\begin{figure}[!htbp]
    \centering
    
    \begin{lstlisting}
        por(; expresion; asignacion;){
            //Codigo a repetir
        }
        //Codigo continua
    \end{lstlisting}
    \caption{Ejemplo de \emph{por} sin la primera asignación}
\end{figure}

\FloatBarrier
\subsection{Funciones}

Otra parte importante de los lenguajes de programación es la posibilidad de poder romper el código en funciones que cumplen un funcionamiento especifico, definido por el programador. IntroProg estructura sus funciones de la siguiente manera.

\subsubsection{Declaración de función}

Antes de utilizar funciones es importante primero que el usuario declare las funciones. Para ello hay que escribir la declaración de las funciones en el espacio apropiado. En la estructura de un programa de IntroProg las funciones se declaran después de las declaraciones de las variables globales, pero antes de la función principal.

\begin{figure}[!htbp]
    \centering
    
    \begin{lstlisting}
        programa pelos{
            // Primero las variables globales
            entero x;
            bool bo;
            // Aqui van las funciones
            funcion vacio holaMundo(){}{
                imprimir("Hola Mundo!!!");
            }
            // El codigo principal del programa
            principal funcion {
                // Variables locales a principal
                entero x,y;
            }{
                x = 10;
                y = 2;
                x = x*y;
                imprimir(x);
                $holaMundo();
            }
        }
    \end{lstlisting}
    \caption{Ejemplo del lugar donde se declaran las funciones}
\end{figure}
\FloatBarrier

El programador puede declarar una función usando la palabra clave \emph{funcion} seguida por el tipo de función. Después se escribe los parámetros de la función, escritos entre paréntesis denotando el tipo y el identificador del parámetro. Es importante remarcar que una función puede no tener parámetros. Se pueden definir múltiples parámetros separando con comas cada uno de ellos. Después de definir los parámetros se escribe entre corchetas las declaraciones de variables. Si no se planea usar variables locales en la función se puede dejar vació el espacio entre los corchetes. Después de declarar las variables se escribe el bloque de código que debe de ejecutar la función entre corchetas.

\begin{figure}[!htbp]
    \centering
    
    \begin{lstlisting}
        funcion vacio holaMundo(){}{
            imprimir("Hola Mundo!!!");
        }
    \end{lstlisting}
    \caption{Ejemplo de función vacia}
\end{figure}
\FloatBarrier

\begin{figure}[!htbp]
    \centering
    
    \begin{lstlisting}
        funcion entero suma( entero n, entero m){}{
            regresa n * m;
        }
    \end{lstlisting}
    \caption{Ejemplo de función con retorno}
\end{figure}
\FloatBarrier

En IntroProg hay dos tipos de funciones, las funciones vacías y las funciones con retorno. Las funciones vacías sirven solo para ejecutar código que no se esperan que regresen un valor, esto puede ser imprimir el resultado de ciertas variables o hacer cálculos temporales que no se van a utilizar después. Las funciones con retorno son funciones que realizan operaciones y que regresan un resultado. El valor de retorno de estas funciones están indicadas por la palabra clave \emph{regresar} seguida por una expresión el cual su resultado es el que va a ser utilizado. Cabe recalcar que las funciones vacío no pueden tener el estatuto de \emph{regresar}.


\subsubsection{Llamada de función}

Una vez declarada las funciones pueden ser utilizadas como el elemento de una expresión. Esto se logra utilizando el símbolo de dólar (\emph{\$}) seguido por el nombre de la función, seguido por un paréntesis con expresiones separados por comas, representando los argumentos. Si la función no tiene argumentos se dejan los paréntesis vacíos.

\begin{figure}[!htbp]
    \centering
   
    \begin{lstlisting}
            principal funcion {
                // Variables locales a principal
                entero x,y;
            }{
                x = 10;
                y = 2;
                x = $suma(x, x*y);
                imprimir(x);
                $holaMundo();
            }
    \end{lstlisting}
     \caption{Ejemplo del lugar donde se declaran las funciones}
\end{figure}
\FloatBarrier
\subsubsection{Llamadas recursivas en funciones}

IntroProg también soporta el uso de funciones recursivas. Esto se puede lograr haciendo la llamada de la misma función dentro de su declaración. Esto puede ser útil para resolver ciertos problemas de programación.

\begin{figure}[!htbp]
    \centering
    
    \begin{lstlisting}
        funcion entera factorial ( entero  n) {}{
            regresa n * $factorial(n-1);
        } 
    \end{lstlisting}
    \caption{Ejemplo de función recursiva}
\end{figure}


\FloatBarrier
\subsection{Arreglos, Matrices y Cubos}

En IntroProg las variables no solo se pueden usar para guardar solo un valor. Se pueden declarar arreglos los cuales pueden tener múltiples valores dependiendo de su declaración. Para declarar un arreglo se usa una sintaxis muy similar a la declaración simple. La declaración de un arreglo se da escribiendo primero el tipo de la variable, seguido por su identificador, seguido del tamaño del arreglo encerrado entre corchetas cuadradas.\\

\begin{figure}[!htbp]
    \centering
    
    \begin{lstlisting}
        entero arreglo [10]; // Un arreglo de 10 valores

    \end{lstlisting}
    \caption{Ejemplo de la declaración de un arreglo}
\end{figure}
\FloatBarrier


Para poder accesar a uno de los valores que tiene el arreglo se puede llamar en una expresión de la siguiente manera. Se escribe el nombre del arreglo seguido de la posición del valor entre corchetas cuadradas. Las posiciones en IntroProg se empiezan a contar desde el 0 hasta el número entero anterior al tamaño.\\

\begin{figure}[!htbp]
    \centering
    
    \begin{lstlisting}
        arreglo[0] = 10; // primera posicion del arreglo
        arreglo[9] = 2003; // Ultima posicion del arreglo
        //Tambien se pueden usar para operaciones
        arreglo[2] = arreglo[0] * arreglo[9];

    \end{lstlisting}
    \caption{Ejemplo de la llamada de un arreglo}
\end{figure}
\FloatBarrier
Los arreglos en IntroProg también se les pueden asignar valores de manera directa a todas las casillas del arreglo. Esto se puede lograr listando los valores entre corchetas cuadradas y separando los valores entre comas.

\begin{figure}[!htbp]
    \centering
    
    \begin{lstlisting}
        arreglo = [1,2,3,4,5,6,7,8,9,10]

    \end{lstlisting}
    \caption{Ejemplo de una asignación textual a un arreglo}
\end{figure}
\FloatBarrier
\subsubsection{Matrices y Cubos}
IntroProg no solo maneja arreglos sino también maneja matrices y cubos. La forma de interactuar con ellos es muy similar a los arreglos. Solo se necesita hacer lo mismo que se hace con un arreglo solo que en vez de utilizar solo un set de corchetas cuadradas se usan la cantidad de corchetas cuadradas a las dimensiones a utilizar. Es importante mencionar que IntroProg solo maneja hasta cubos y no permite crear variables con más dimensiones.

\begin{figure}[!htbp]
    \centering
    
    \begin{lstlisting}
        entero mat[3][3];
        entero cubo[2][2][2];

    \end{lstlisting}
    \caption{Ejemplo de la declaración de una matriz y un cubo}
\end{figure}
\FloatBarrier

\begin{figure}[!htbp]
    \centering
    
    \begin{lstlisting}
        mat[0][0] = 10;
        cubo[0][1][1] = 10;

    \end{lstlisting}
    \caption{Ejemplo de la llamada de una matriz y un cubo}
\end{figure}
\FloatBarrier

\begin{figure}[!htbp]
    \centering
    
    \begin{lstlisting}
        mat = [[1,2,3],[4,5,6],[7,8,9]];
        cubo = [[[1,2],[3,4]],[[5,6],[7,8]]];

    \end{lstlisting}
    \caption{Ejemplo de una asignación textual de una matriz y un cubo}
\end{figure}
\FloatBarrier






\FloatBarrier
\subsection{Funciones Especiales}

IntroProg también cuenta con una serie de funciones especiales que pueden ser utilizadas en cualquier programa. Estas funciones tienen un enfoque hacia estadística, y le permiten a los usuarios hacer programas más complejos. Estas funciones son:

\begin{itemize}
    \item \$leer() : Regresa la lectura capturada de consola
    \item \$modulo(flotante a, flotante b): Regresa el resultado de a\%b
    \item \$suma(flotante a[]): Suma todos los elementos de un arreglo y regresa sus resultados
    \item \$raiz(flotante a): Regresa el flotante resultante de la raíz de a
    \item \$exp(flotante a): Regresa la exponencial de e\^a
    \item \$elevar(flotante a, flotante b): Regresa el resultado de a\^b
    \item \$techo(flotante a): Regresa un flotante a redondeado para arriba
    \item \$piso(flotante a): Regresa un flotante a redondeado para abajo
    \item \$cos(flotante a): Regresa el coseno de a
    \item \$sen(flotante a): Regresa el seno de a
    \item \$tan(flotante a): Regresa la tangente de a
    \item \$cotan(flotante a): Regresa la cotangente de a
    \item \$sec(flotante a): Regresa la secante de a
    \item \$cosec(flotante a): Regresa la cosecante de a
    \item \$log(flotante a): Regresa el logaritmo natural de a
    \item \$minimo(flotante a[]): Regresa el valor más chico en el vector a
    \item \$maximo(flotante a[]): Regresa el valor máximo de a
    \item \$redondear(flotante a): Regresa un flotante a redondeado 
    \item \$productoPunto(flotante a[], flotante b[]): Regresa el producto punto entre los vectores de entrada a y b
    \item \$media(flotante a[]): Regresa la media de a
    \item \$mediana(flotante a[]): regresa la mediana de a
    \item \$moda(flotante a[]): Regresa el elemento con la moda más alta de a
    \item \$varianza(a): Regresa la varianza de a
    \item \$percentil(flotante a[], flotante q): Regresa el valor en el que se encuentran q% de los valores en a
    \item \$aleatorio(flotante min, flotante max): Regresa un número flotante aleatorio entre los rangos de argumentos mínimos y máximos
    \item \$wilcoxon(flotante x[]): Realiza la prueba de Wilcoxon en la serie de datos en x
    \item \$wilcoxonComp(flotante x[], flotante y[]):Realiza la prueba de Wilcoxon se realiza la prueba sobre los datos x y y.
    \item \$regresionSimple(flotante x[], flotante y[], flotante xi): Dado un set de x y y se usará regresión lineal simple para encontrar f(xi) y se regresara ese valor
    \item \$normal(flotante media, flotante desv): Regresa un número escalar que pertenezca a la distribución normal dado los parámetros
    \item \$poisson(flotante lambda): Regresa un número aleatorio de la distribución Poisson con la lambda dada
    \item \$dexponencial(flotante beta): Regresa un número aleatorio de la distribución Exponencial correspondiente a la beta (o 1/Lambda) dada.
    \item \$dgeometrica(flotante exito): Te regresa un valor con la distribución geométrica con la probabilidad de exito dada. 
    \item \$histograma(flotante x[], flotante rango): Genera un histograma a partir de los datos en el vector de x con un rango entre los datos de rango.
    \item \$diagramadecaja(flotante x[]): Genera un diagrama de caja y bigotes de los datos en en x
    \item \$grafDispersion(flotante x[], flotante y[]): Genera un gráfico de dispersión con los valores de ‘x’ y ‘y’
\end{itemize}

Para llamar a una función especial se usa la misma sintaxis que una función normal.

\begin{figure}[!htbp]
    \centering
    
    \begin{lstlisting}
            x = $aleatorio(1,100);
    \end{lstlisting}
    \caption{Ejemplo del uso de funcinoes especiales}
\end{figure}
\FloatBarrier

\FloatBarrier

%---------------------------------------------------------------------------------------

\section{Listado de errores}
El compilador y la máquina virtual manejan distintos tipos de errores que pueden suceder cuando se hace un programa. Como el lenguaje está diseñado para ser un punto de entrada a la programación se intentó cubrir la mayor cantidad de errores de programación comunes y se agregaron mensajes de error apropiados a la situación del error. Estos mensajes de error tratan de explicar de una manera clara cuál es la situación que causo el error al usuario.\\

Los errores que se pueden detectar durante compilación son los siguientes:
\begin{itemize}
    \item Doble declaración de variables
    \item El uso de variables no declaradas.
    \item Errores de sintaxis. Esto incluye identificadores y palabras reservadas mal escritas, estatutos mal escritos entre otros errores de sintaxis.
    \item Número de parámetros incorrectos
    \item Asignación incorrecta de parámetros, si el valor pasado como parámetro no se puede convertir a al tipo del parámetro el compilador levanta un error.
    \item Operaciones con funciones vacías, el compilador no permite el uso de funciones vacías en expresiones
    \item Retornos en funciones vacías, El compilador levanta errores cuando detecta un retorno cuando hay retornos en funciones vacías
    \item No permitir el nombre de funciones y palabras reservadas como identificadores.
\end{itemize}

Los errores que puede detectar la máquina virtual son la siguiente:
\begin{itemize}
    \item Operaciones con variables sin valor. Si se intenta realizar operaciones con variables las cuales no cuentan con un valor se levanta un error y se le indica al usuario cuál fue el posible error.
    \item Detectar si los subíndices del arreglo se salen del espacio definido por el arreglo. Similarmente detecta cuando el arreglo se sale de los límites del arreglo e indica al usuario mediante un mensaje de error lo que sucedió.
    \item Lectura incorrecta. Cuando se utiliza la lectura y se mete un valor que no es válido para lectura.
\end{itemize}




\part{Descripción del compilador y la máquina virtual}
\chapter{Descripción del compilador}

En esta parte se describe con más detalle el compilador

\section{Equipo de cómputo, lenguaje y utilerías especiales usadas en el desarrollo del proyecto}

Una PC con Windows 10, se utilizó el lenguaje de programación Python 3.10 con apoyo de las librerías de \emph{PLY, Numpy, re, json}.

Para el compilador se utilizó la librería de \emph{Ply} para el manejo del análisis léxico, sintáctico, y la semántica estática del lenguaje. En el caso del análisis semántico también se utilizó la librería de \emph{re} para manejar expresiones regulares en la lógica de semántica y código intermedio. Finalmente para poder guardar con mayor facilidad el código intermedio, la tabla de funciones y la tabla de constantes se uso la librería de JSON para exportarlos a un archivo.


\FloatBarrier
%---------------------------------------------------------------------------------------
\section{Descripción del Análisis de Léxico}

Para este lenguaje se opto por utilizar palabras clave en español. Esto se decidió con el objetivo de ser más intuitivo para las personas que hablan español y que tengan poco conocimiento del lenguaje inglés.

En IntroProg se usaron las siguientes Palabras reservadas:
\begin{itemize}
    \item programa
    \item funcion
    \item si
    \item sino
    \item mientras
    \item por
    \item entero
    \item flotante
    \item char
    \item cadena
    \item bool
    \item vacío
    \item nulo
    \item verdadero
    \item falso
    \item principal
    \item imprimir
    \item regresar
\end{itemize}

A su vez los tokens importantes del lenguaje son los siguientes:

\begin{itemize}
    \item dígito : $[0-9]$
    \item letra : $[a-zA-Z]$
    \item CTE\_INT : dígito$+$
    \item CTE\_FLOAT : dígito$+$\textbackslash .dígito$+$
    \item CTE\_STRING : \textbackslash” [ \textbackslash \textasciicircum  ” ] \textbackslash”
    \item COMMENT : \textbackslash /\textbackslash /.*
    \item espacio blanco : [ \textbackslash n\textbackslash t]$+$
    
    \item ID : letra(letra | dígito | \_ )*
    \item SEMICOLON : $;$
    \item COMMA : $,$
    \item EQ : $=$
    \item OPENPAR : \textbackslash$($
    \item CLOSEPAR : \textbackslash$)$
    \item GT : $<$
    \item LT : $>$
    \item PLUS : $+$
    \item MINUS : $-$
    \item MUL : $*$
    \item DIV : $/$
    \item OPENCUR :\textbackslash $ \{ $
    \item CLOSECUR : \textbackslash $\}$
    \item OPENSQU : \textbackslash $[$
    \item CLOSESQU : \textbackslash $]$
    \item AND : $\&\&$
    \item OR : $||$
    \item EXLAM : $!$
    \item DLR : $\$$
\end{itemize}


\FloatBarrier
%---------------------------------------------------------------------------------------
\section{Descripción del Análisis de Sintaxis}

%------------------GRAMATICA-----------------------------------------------
Con los tokens definidos en la parte léxica se desarrolló la siguiente gramática para describir el lenguaje :

{\small
\begin{enumerate}
    \item PROGRAM → PROGRAMA ID OPENCUR DECLARACIONES DECTODASFUNC PRINCIPAL FUNCION OPENCUR DECLARACIONES CLOSECUR BLOQUE CLOSECUR
    \item DECLARACIONES → DECLARACION DECLARACIONES \\ | $\epsilon$
    \item DECLARACION → DECVAR \\ \quad | DECARR
    \item DECVAR → TIPO ID DVNID SEMICOLON
    \item DVNID → COMMA ID DVNID \\ | $\epsilon$
    \item DECARR → TIPO ID OPENSQU CTE\_INT CLOSESQU DSEG SEMICOLON
    \item DSEG → OPENSQU CTE\_INT CLOSESQU DTER \\ | $\epsilon$
    \item DTER → OPENSQU CTE\_INT CLOSESQU \\ | $\epsilon$
    \item TIPO → ENTERO \\ | FLOTANTE \\ | CHAR \\ | BOOL
    \item DECTODASFUNC \\ | DECFUNC DECTODASFUNC
    \item DECFUNC →  FUNCION TIPOFUN ID OPENPAR FUNPARAM CLOSEPAR OPENCUR DECLARACIONES CLOSECUR BLOQUE
    \item TIPOFUN → TIPO \\ | VACIO
    \item FUNPARAM → PARAM \\ | $\epsilon$
    \item PARAM → TIPO ID PARAMD PARAMS
    \item PARAMS → COMMA PARAM \\ | $\epsilon$
    \item PARAMD → OPENSQU CTE\_INT CLOSESQU PDSEG \\ | $\epsilon$
    \item PDSEG → OPENSQU CTE\_INT CLOSESQU PDTER \\ | $\epsilon$
    \item PDTER → OPENSQU CTE\_INT CLOSESQU \\ | $\epsilon$
    \item BLOQUE → OPENCUR ESTATUTOS CLOSECUR
    \item ESTATUTOS → ESTATUTO ESTATUTOS \\ | $\epsilon$
    \item ESTATUTO → IMPRESION \\ | ASIGNACION \\ | EXPRESION SEMICOLON \\ | CONDICION \\ | BUCLE \\ | RETURNF
    \item IMPRESION → IMPRIMIR OPENPAR PRINTABLE PRINTARGS CLOSEPAR SEMICOLON                   
    \item PRINTARGS → COMMA PRINTABLE PRINTARGS \\ | $\epsilon$
    \item PRINTABLE → EXPRESION \\ | CTE\_STRING \\ |  $\epsilon$   
    \item ASIGNACION → ID ADIMS EQ EXPRESION  SEMICOLON \\ | ID EQ ARR\_TEX SEMICOLON
    \item ADIMS → OPENSQU EXPRESION CLOSESQU ASEGD \\ | $\epsilon$
    \item ASEGD → OPENSQU EXPRESION CLOSESQU ATERD \\ | $\epsilon$             
    \item ATERD → OPENSQU EXPRESION CLOSESQU \\ | $\epsilon$
    \item CONDICION → SI OPENPAR EXPRESION CLOSEPAR BLOQUE IFELSE
    \item IFELSE → SINO BLOQUE \\ | $\epsilon$
    \item BUCLE → WHILE \\ | FOR
    \item WHILE → MIENTRAS OPENPAR EXPRESION CLOSEPAR BLOQUE
    \item FOR → POR OPENPAR FORINIT EXPRESION SEMICOLON ASIGNACION CLOSEPAR BLOQUE
    \item FORINIT → ASIGNACION \\ | $\epsilon$
    \item RETURNF → REGRESAR EXPRESION SEMICOLON
    \item EXPRESION → EXPRESIONR \\ | EXPRLOG
    \item EXPRLOG → EXPRESION AND EXPRESION \\ | EXPRESION OR EXPRESION \\ | $\epsilon$
    \item EXPRESIONR → EXP \\ | EXPR
    \item EXPR : EXP LT EXP \\ | EXP GT EXP \\ | EXP EXLAM EQ EXP \\ | EXP EQ EQ EXP \\ | EXP LT EQ EXP \\ | EXP GT EQ EXP \\ | $\epsilon$
    \item EXP → TERMINO \\ | TERMINOSS
    \item TERMINOSS → EXP PLUS EXP \\ | EXP MINUS EXP \\ | $\epsilon$
    \item TERMINO → FACTOR \\ | FACTORESS
    \item FACTORES → TERMINO MUL TERMINO \\ | TERMINO DIV TERMINO \\ | $\epsilon$
    \item FACTOR → SIGNOVAR VARCTE \\ | OPENPAR EXPRESION CLOSEPAR
    \item SIGNOVAR → PLUS \\ | MINUS \\ | $\epsilon$
    \item VARCTE : ID \\ | CTE\_INT \\ | CTE\_FLOAT \\ | CTE\_STRING \\ | CTE\_BOOL \\ | CTE\_CHAR \\ | LLAMADAFUNC \\ | LLAMADAARR \\ | NULO
    \item LLAMADAFUNC →  DLR ID OPENPAR CALLPARAMS CLOSEPAR
    \item CALLPARAMS → CPARAM \\ | $\epsilon$
    \item CPARAM → EXPRESION CPARAMS
    \item CPARAMS → COMMA EXPRESION CPARAMS \\ | $\epsilon$
    \item LLAMADAARR → ID OPENSQU EXPRESION CLOSESQU LLSEGD
    \item LLSEGD → OPENSQU EXPRESION CLOSESQU LLTERD \\ | $\epsilon$
    \item LLTERD → OPENSQU EXPRESION CLOSESQU \\ | $\epsilon$
    \item ARR\_TEX → OPENSQU ATPRIC CLOSESQU
    \item ATPRIC → ATPRE ATPRISIG \\ | $\epsilon$
    \item ATPRE → EXPRESION \\ | ATSEGD
    \item ATPRISIG → COMMA ATPRE ATPRISIG \\ | $\epsilon$
    \item ATSEGD → OPENSQU ATSEGC CLOSESQU
    \item ATSEGC → ATSEGE ATSEGSIG \\ | $\epsilon$
    \item ATSEGE → EXPRESION \\ | ATTERD
    \item ATSEGSIG → COMMA ATSEGE ATSEGSIG \\ | $\epsilon$
    \item ATTERD → OPENSQU ATTERC CLOSESQU
    \item ATTERC → ATTERE ATTERSIG \\ | $\epsilon$
    \item ATTERE → EXPRESION
    \item ATTERSIG → COMMA ATTERE ATTERSIG \\ | $\epsilon$
    
\end{enumerate}}

Con esta gramática se procedió a generar diagramas sintácticos.


\FloatBarrier
\newpage
%---------------------------------------------------------------------------------------
\section{Descripción de Generación de Código Intermedio y Análisis Semántico}

En esta sección se explica el funcionamiento del Código Intermedio y el funcionamiento del Análisis semántico.
\subsubsection{Direcciones de Memoria Virtuales}
El compilador en todas sus operaciones asume una abstracción del manejo de memoria en la máquina virtual. Para lograr esto el compilador asigna direcciones de memoria virtuales a las constantes y variables que encuentra en su ejecución. Estas direcciones están son las siguientes.

\begin{itemize}
    \item Direcciones de las variables Globales \begin{itemize}
        \item Enteros : 1000
        \item Flotantes: 3000
        \item Caracteres :5000
        \item Booleanos: 6000
        \item Apuntadores: 7000
    \end{itemize}
    
    \item Direcciones de las Variables Locales \begin{itemize}
        \item Enteros : 9000
        \item Flotantes : 11 000
        \item Caracteres : 13 000
        \item Booleanos : 14 000
        \item Apuntadores : 15 000
    \end{itemize}
    
    \item Direcciones de las Variables Temporales \begin{itemize}
        \item Enteros: 17 000
        \item Flotantes: 19 000
        \item Caracteres: 21 000
        \item Booleanos: 22 000
        \item Apuntadores:  23 000
    \end{itemize}
    
    \item Direcciones de las Constantes \begin{itemize}
        \item Enteros: 25 000
        \item Flotantes: 27 000
        \item Caracteres: 29 000
        \item Booleanos: 30 000
        \item Cadenas:  31 000
    \end{itemize}
\end{itemize}

No solo asigna direcciones virtuales también mantiene un conteo de los recursos que se están utilizando. Si se pasa de la cantidad de variables de cada tipo y scope que se pueden usar en un espacio de memoria el compilador va a levantar un error. Estos límites son los siguientes:

\begin{itemize}
    \item Máximo de Enteros = 2000
    \item Máximo de Flotantes = 2000
    \item Máximo de Caracteres = 1000
    \item Máximo de Booleanos = 1000
    \item Máximo de Apuntadores = 2000
    \item Máximo de Cadenas = 1000
\end{itemize}

Con estos máximos se puede detectar si el usuario está declarando variables demás. Cada uno de estos máximos se dan por scope, es decir un programa puede tener un máximo de 2000 enteros globales y 2000 enteros locales en cada función. Si el usuario intenta declarar más variables que esas el compilador va a levantar un error indicando que se sobre paso el número de variables permitidos.

\subsubsection{Códigos Intermedios (Cuadruplos)}

Para el código intermedio del compilador se pensaron en las acciones básicas que debe de hacer la máquina virtual. Para indicar estas acciones se utilizaron el formato de cuádruplos para representar estos códigos. Teniendo claro esto, se definieron cuáles son las acciones más importantes que se necesitan para traducir el lenguaje a operaciones que la máquina virtual pueda interpretar con facilidad. Las acciones que son de interés para que la máquina virtual las ejecute son principalmente acciones de expresiones, asignación, saltos, manejo de memoria y código de funciones especial. Con esto en mente se desarrollaron los siguientes códigos:

\begin{itemize}
    \item \emph{Códigos de Expresiones y Asignación:} Son códigos de operación que se enfocan en realizar operaciones de expresión. Estos están formados por el operador, el cual puedes ser cualquiera de los operadores de expresiones ($+$,$-$,*, /, etc), seguidos por el operando izquierdo, seguido por el operando derecho y en la cuarta posición se la dirección virtual del resultado. Solo existe un caso especial con expresiones, el cual es cuando el operador es un $+$ o un $-$, el cual puede recibir un solo operando y este representa cuando se definen números positivos y negativos.
    
    % Ejemplos de cuadruplos de expresiones
    \begin{figure}[!htbp]
        \centering
        \begin{lstlisting}
           ["/", 17000, 25007, 17001]
           ["*", 17001, 25008, 17002]
           ["+", 25006, 17002, 17003]
           ["-", 25001, 25009, 17004]
           ["<", 17007, 25008, 22001]
           [">", 17007, 25008, 22001]
           [">=", 17007, 25008, 22001]
           ["<=", 17007, 25008, 22001]
           ["==", 17007, 25008, 22001]
           ["!=", 17007, 25008, 22001]
           ["||", 22000, 22001, 22002]
           ["&&", 22002, 30000, 22003]
           ["=", 22003, "", 14000]
        \end{lstlisting}
        \caption{Ejemplos de códigos Expresiones y asignación}
        \label{fig:my_label}
    \end{figure}
    \FloatBarrier
    
    \item \emph{Código de Saltos:} Son códigos que expresan un salto a otra instrucción. Existen dos tipos de este código los cuales son GOTO y GOTOF. El primero está conformado por la palabra clave GOTO en la primera posición y en la última posición el número del cuádruplo a donde saltar. El segundo código tiene un comportamiento diferente al primero. Este código se usa para indicar que se debe de hacer el salto si el valor en una dirección es falsa. La Composición de este cuádruplo estará dada por la palabra GOTOF en la primera posición, la dirección a evaluar en la segunda y en la última posición el número del cuádruplo a donde hacer el salto si la expresión es falsa.
    
    % Ejemplos de cuadruplos de saltos
    \begin{figure}[!htbp]
    \centering
    \begin{lstlisting}
       [ 'GOTOF' 22003 '' 112 ]
       [ 'GOTO' '' '' 91 ]
    \end{lstlisting}
    \caption{Ejemplos de códigos de saltos}
    \label{fig:my_label}
\end{figure}
\FloatBarrier
    
    \item \emph{Código de Funciones:} Son cuádruplos que indican las acciones necesarias para ejecutar funciones. Para las funciones se crear cuatro códigos para indicar su comportamiento. Estos son ERA, PARAMETER, GOTOSUB, ENDFUNC y SPFUNC. ERA es un código que indica la creación de un nuevo espacio de memoria. Este se genera escribiendo el código ERA en la primera posición y en la segunda el nombre de la función. El segundo código, PARAMETER, es un código utilizado para indicar que se debe de copiar la información de una variable a el nuevo espacio de memoria generado. Este código se escribe en un cuádruplo como la palabra clave PARAMETER en la primera posición, seguida por la dirección a copiar en la segunda posición. La tercera posición se puede dejar vacía si es una variable simple, pero si es un arreglo se escribe el tamaño en ella. Y en la última posición se escribe el tipo del parámetro y el número del parámetro separados por el símbolo \#. El tercer código es GOTOSUB. Este código sirve para indicar el cambio de contexto a una función. Este código se escribe poniendo la palabra clave GOTOSUB en la primera posición, el nombre de la función en la segunda posición. En la última posición se escribe el número del cuádruplo donde empieza la función. El cuarto código ENDFUNC sirve para indicar el fin de una función y el regreso al contexto anterior de la memoria. Este cuádruplo es muy sencillo y consta solo de la palabra clave ENDFUC en la primera posición. Finalmente se tiene el código SPFUNC. Este código se utiliza para indicar el uso de una función especial, en vez de una función del programa. Este código se genera de la siguiente manera. En la primera posición se escribe la palabra clave SPFUNC, en la segunda se escribe el nombre de la función, en la tercera posición el tipo de la función y en la última posición se escribe la dirección de la variable global de retorno si es una función que regresa un valor, sino se deja vacío.
    
    % Ejemplos de cuadruplos de funciones
\begin{figure}[!htbp]
    \centering
    \begin{lstlisting}
        ["PARAMETER", 25001, "", "1#1"]
        ["ERA", "pop", "", ""]
        ["GOTOSUB", "pop", "", 1]
        ["ENDFUNC" , "", "", ""]
        ["SPFUNC", "dexponencial", 1, 3026]
        ["SPFUNC", "diagramaDeCaja", -1, ""]
    \end{lstlisting}
    \caption{Ejemplos de códigos de funciones}
    \label{fig:my_label}
\end{figure}
\FloatBarrier
\end{itemize}

\subsubsection{Cubo Semántico}
Para el uso del código intermedio para expresiones se definio como interactuan los tipos de las variables cuando se realizan cada tipo de operación. Esto esta dado por la siguiente tabla.

\begin{table}
    \centering
    \caption{Tabla representando el cubo semántico}
    \small
    \begin{tabular}{||c c || c c c c c c c c c c c c c||} 
         
         OpIzq & OpDer & $=$ & $\parallel$ & \&\& & \textless & \textgreater & \textless $=$ & \textgreater $=$ & !$=$ & $==$ & $+$ & $-$ & * & \textbackslash \\ [0.5ex] 
         \hline\hline
         ent & ent & ent & bool & bool & bool & bool & bool & bool & bool & bool & ent & ent & ent & ent \\ 
         \hline
         ent & flot & ent & bool & bool & bool & bool & bool & bool & bool & bool & flot & flot & flot & flot \\ 
         \hline
         ent & char & ent & bool & bool & bool & bool & bool & bool & bool & bool & ent & ent & ent & ent \\ 
         \hline
         ent & bool & ent & bool & bool & bool & bool & bool & bool & bool & bool & ent & ent & ent & ent \\ 
         \hline
         ent & cadena & err & err & err & err & err & err & err & err & err & err & err & err & err \\ 
         \hline
         flot & ent & flot & bool & bool & bool & bool & bool & bool & bool & bool & flot & flot & flot & flot \\ 
         \hline
         flot & flot & flot & bool & bool & bool & bool & bool & bool & bool & bool & flot & flot & flot & flot \\ 
         \hline
         flot & char & flot & bool & bool & bool & bool & bool & bool & bool & bool & flot & flot & flot & flot \\ 
         \hline
         flot & bool & flot & bool & bool & bool & bool & bool & bool & bool & bool & flot & flot & flot & flot \\ 
         \hline
         flot & cadena & err & err & err & err & err & err & err & err & err & err & err & err & err \\ 
         \hline
         char & ent & char & bool & bool & bool & bool & bool & bool & bool & bool & ent & ent & ent & ent \\ 
         \hline
         char & flot & char & bool & bool & bool & bool & bool & bool & bool & bool & flot & flot & flot & flot \\ 
         \hline
         char & char & char & bool & bool & bool & bool & bool & bool & bool & bool & ent & ent & ent & ent \\ 
         \hline
         char & bool & char & bool & bool & bool & bool & bool & bool & bool & bool & ent & ent & ent & ent \\ 
         \hline
         char & cadena & err & err & err & err & err & err & err & err & err & err & err & err & err \\ 
         \hline
         bool & ent & bool & bool & bool & bool & bool & bool & bool & bool & bool & ent & ent & ent & ent \\ 
         \hline
         bool & flot & bool & bool & bool & bool & bool & bool & bool & bool & bool & flot & flot & flot & flot \\ 
         \hline
         bool & char & bool & bool & bool & bool & bool & bool & bool & bool & bool & ent & ent & ent & ent \\ 
         \hline
         bool & bool & bool & bool & bool & bool & bool & bool & bool & bool & bool & ent & ent & ent & ent \\ 
         \hline
         bool & cadena & err & err & err & err & err & err & err & err & err & err & err & err & err \\ 
         \hline
         cadena & ent & err & err & err & err & err & err & err & err & err & err & err & err & err \\ 
         \hline
         cadena & flot & err & err & err & err & err & err & err & err & err & err & err & err & err \\ 
         \hline
         cadena & char & err & err & err & err & err & err & err & err & err & err & err & err & err \\ 
         \hline
         cadena & bool & err & err & err & err & err & err & err & err & err & err & err & err & err \\ 
         \hline
         cadena & err & err & err & err & err & err & err & err & err & err & err & err & err & err \\ 
        
         
        
        \end{tabular}
        \begin{itemize}
            \item ent : Entero
            \item flot : Flotante
            \item err : Error
        \end{itemize}
    
\end{table}
\FloatBarrier

\subsubsection{Diagramas de Sintaxis y Acciones Semánticas}
A Continuación, se demuestran los diagramas sintácticos con sus acciones semánticas correspondientes.

\begin{enumerate}
    \begin{figure}[!htbp]
            \centering
            \includegraphics[width=\textwidth]{chapters/chapter3/figures/diagramas compis-Programa.drawio(1).png}
            \caption{Diagrama Programa}
            \label{fig:my_label}
    \end{figure}
    \FloatBarrier
    \item Se genera el cuádruplo de goto main y se mete el valor de contador de cuadruplos a la pila de saltos. Se actualiza la variable global del scope como global
    \item Se agrega la entrada temporal de la variable a la tabla de variables en la entrada del espacio global en el directorio de funciones.
    \item Se itera por la tabla de variables y se le asigna a cada variable una dirección virtual. Al iterar por la tabla también se actualiza el contador de recursos de las variables globales
    \item Se agrega la función a el directorio de variables
    \item Se genera la tabla de variables para el scope principal y se itera por la tabla asignando las direcciones virtuales adecuadas. Al terminar se hace pop a la pila de saltos y se actualiza el primer cuádruplo con el contador actual de cuádruplos.
    \newpage
    
    %Dec
    \begin{figure}[!htbp]
            \centering
            \includegraphics[width=\textwidth]{chapters/chapter3/figures/diagramas compis-declaración.drawio(1).png}
            \caption{Diagrama Declaración}
            \label{fig:my_label}
    \end{figure}
    \FloatBarrier
    % Tipo
    \begin{figure}[!htbp]
            \centering
            \includegraphics[width=\textwidth]{chapters/chapter3/figures/diagramas compis-tipo.drawio.png}
            \caption{Diagrama Tipo}
            \label{fig:my_label}
    \end{figure}
    \FloatBarrier
    
    %Dec de variable
    \begin{figure}[!htbp]
            \centering
            \includegraphics[width=\textwidth]{chapters/chapter3/figures/diagramas compis-declaración variable.drawio(1).png}
            \caption{Diagrama Declaración Variable}
            \label{fig:my_label}
    \end{figure}
    \FloatBarrier
    
    \item Se guarda en una variable auxiliar global el tipo de variable
    \item Se checa que el id no se repita con otros ids del scope de la declaración. Si no se repite se genera una nueva entrada de variable con el nuevo id con el tipo de variable dado por el punto 6
    
    \newpage
    %DEc de arreglo
    
    \begin{figure}[!htbp]
            \centering
            \includegraphics[width=\textwidth]{chapters/chapter3/figures/diagramas compis-declaración arreglo.drawio(1).png}
            \caption{Diagrama Declaración Arreglo}
            \label{fig:my_label}
    \end{figure}
    \FloatBarrier
    
    \item Se agrega la información de la primera dimensión a la entrada de la variable
    \item Se agrega la información de la segunda dimensión a la entrada de la variable
    \item Se agrega la información de la tercera dimensión a la entrada de la variable
    \item Se hace el cálculo de las m y el tamaño del arreglo/matriz/cubo (depende de las dimensiones leídas) y se actualiza la entrada
    
    \newpage
    % Dec Func
    
    \begin{figure}[!htbp]
            \centering
            \includegraphics[width=\textwidth]{chapters/chapter3/figures/diagramas compis-dec_funcion.drawio(1).png}
            \caption{Diagrama Declaración Función}
            \label{fig:my_label}
    \end{figure}
    \FloatBarrier
    
    
    \item Se guarda en una variable auxiliar global el número del cuádruplo donde empieza la función y se genera una entrada temporal al directorio de funciones.
    \item Se guarda el tipo de función a la entrada temporal
    \item Se comprueba que el identificador no sea igual a otros identificadores, palabras clave o nombres de funciones especiales. Si no hay conflictos se agrega el nombre a la entrada temporal, si el tipo no es vacío, genera una variable global con el nombre y tipo de la función con la dirección virtual apropiada y se actualiza la variable global de scope y los contadores de recursos globales.
    \item Se checa que el id no se repita entre otros parámetros. Si no se repite se agrega el parámetro con el id y el tipo detectado a la sección de parámetros de la entrada temporal
    \item Se agrega la información de la primera dimensión a la entrada del parámetro
    \item Se agrega la información de la segunda dimensión a la entrada del parámetro
    \item Se agrega la información de la tercera dimensión a la entrada del parámetro
    \item Se hace el cálculo de las m y el tamaño del arreglo/matriz/cubo (depende de las dimensiones leídas)y se actualiza la entrada
    \item Se crean entradas a la tabla de variables locales de la funciones con la información de los parámetros
    \item Se agrega la entrada temporal de la tabla de variables a la tabla de variables local.
    \item Se itera por la tabla de variables y se le asigna a cada variable una dirección virtual.
    \item Se genera el cuádruplo de ENDFUNC y se agregan la cantidad de recursos utilizados por la función a la entrada de la función.
    
    \newpage
    
    %Bloque
    
    \begin{figure}[!htbp]
            \centering
            \includegraphics[width=\textwidth]{chapters/chapter3/figures/diagramas compis-bloque.drawio(1).png}
            \caption{Diagrama Bloque}
            \label{fig:my_label}
    \end{figure}
    \FloatBarrier
    
    %Estatuto
    \begin{figure}[!htbp]
            \centering
            \includegraphics[width=\textwidth]{chapters/chapter3/figures/diagramas compis-estatuto.drawio(1).png}
            \caption{Diagrama Estatutos}
            \label{fig:my_label}
    \end{figure}
    \FloatBarrier
    %Return
    \begin{figure}[!htbp]
            \centering
            \includegraphics[width=\textwidth]{chapters/chapter3/figures/diagramas compis-returnf.drawio.png}
            \caption{Diagrama Return}
            \label{fig:my_label}
    \end{figure}
    \FloatBarrier
    \item Checar si la función que llamo a regresar no sea una función vacía. Si es vacia levantar un error
    \item Generar el cuádruplo de [RET, Dirección del resultado de la expresión, , Dirección de la variable global]
    
    \newpage
    
    %Condición
    \begin{figure}[!htbp]
            \centering
            \includegraphics[width=\textwidth]{chapters/chapter3/figures/diagramas compis-condicion.drawio(1).png}
            \caption{Diagrama Condición}
            \label{fig:my_label}
    \end{figure}
    \FloatBarrier
    
    \item Se genera el cuádruplo de gotof con el resultado de la expresión y se agrega número del cuádruplo generado a la pila de saltos.
    \item Se hace pop a la pila de saltos y se genera un cuádruplo de goto y se agrega su número de cuádruplo a la pila. Se actualiza el gotof con el contador actual de cuádruplos
    \item Se hace pop a la pila de saltos y se actualiza el cuádruplo indicado por la pila con el contador actual.
    
    \newpage
    
    %Bucle
    
    \begin{figure}[!htbp]
            \centering
            \includegraphics[width=\textwidth]{chapters/chapter3/figures/diagramas compis-bucle.drawio(1).png}
            \caption{Diagrama Bucle}
            \label{fig:my_label}
    \end{figure}
    \FloatBarrier
    
    %While
    \begin{figure}[!htbp]
            \centering
            \includegraphics[width=\textwidth]{chapters/chapter3/figures/diagramas compis-while.drawio(1).png}
            \caption{Diagrama While}
            \label{fig:my_label}
    \end{figure}
    \FloatBarrier
    
    \item Se mete el contador actual de cuádruplos a la pila de operandos
    \item Se genera un cuádruplo de gotof con el resultado de la expresión y se agrega su número de contador a la pila de saltos
    \item Se le da pop dos veces a la pila de saltos y se genera un cuádruplo de goto con el segundo valor obtenido. Con el primer valor se obtiene el cuádruplo en donde esta el gotof y se actualiza con el contador actual de cuádruplos.
    
    \newpage
    %For
    \begin{figure}[!htbp]
            \centering
            \includegraphics[width=\textwidth]{chapters/chapter3/figures/diagramas compis-for.drawio(1).png}
            \caption{Diagrama For}
            \label{fig:my_label}
    \end{figure}
    \FloatBarrier
    
    \item Se agrega el contador actual a la pila de saltos.
    \item Se checa que la asignación sea una asignación entera
    \item Se genera un gotof con la expresión y se agrega el número de cuádruplo del gotof a la pila de saltos. También se genera un cuádruplo de goto y también se agrega su número de cuádruplo a la pila de saltos.
    \item Se le hace pop a los primeros cuatro elementos de la pila de saltos. Se genera un goto al cuádruplo donde inicia la condición del for, se actualiza el cuádruplo del goto que va al bloque después de la evaluación y se vuelven a insertar los elementos sobrantes.
    \item Se le hace pop a dos elementos de la pila de operandos. Se genera un cuádruplo de goto a donde empieza la asignación de paso del for y se actualiza el cuádruplo de gotof de la condición del for.
    
    \newpage
    
    %Imprimir
    \begin{figure}[!htbp]
            \centering
            \includegraphics[width=\textwidth]{chapters/chapter3/figures/diagramas compis-imprimir.drawio(1).png}
            \caption{Diagrama Imprimir}
            \label{fig:my_label}
    \end{figure}
    \FloatBarrier
    
    \item Se le hace pop a la pila de operandos y se agrega a la fila de impresiones
    \item Se checa que la constante esta en la tabla de constantes, si no esta se agrega asignando la dirección virtual apropiada. Se agrega la dirección virtual a la pila de impresión.
    \item Se recorre la fila de impresión y se va generando un cuádruplo de imprimir con la dirección virtual en la fila
    
    \newpage
    
    %Asignación
    \begin{figure}[!htbp]
            \centering
            \includegraphics[width=\textwidth]{chapters/chapter3/figures/diagramas compis-asignacion.drawio(1).png}
            \caption{Diagrama Asignación}
            \label{fig:my_label}
    \end{figure}
    \FloatBarrier
    
    \item Se checa que el identificador exista en el scope local o en scope global
    \item Se hace pop a la pila de operandos y se revisa que el elemento recibido sea entero. Si es entero se agrega a la fila de subíndices.
    \item Se hace pop a la pila de operandos y se revisa que el elemento recibido sea entero. Si es entero se agrega a la fila de subíndices.
    \item Se hace pop a la pila de operandos y se revisa que el elemento recibido sea entero. Si es entero se agrega a la fila de subíndices.
    \item Se revisa si el identificador es un arreglo. Si es un arreglo se checa que la cantidad de subíndices leídos coincida con las dimensiones del arreglo, de ser así itera por los subíndices generando los cuádruplos de verificación y las sumas apropiadas para calcular la dirección virtual del arreglo accesado.
    \item Se checa contra el cubo semántico si se puede hacer la asignación. De ser posible se genera el cuádruplo de asignación apropiado.
    \item Se checa que el identificador no sea una variable simple o una llamada de arreglo. También se checa que las dimensiones del arreglo textual son iguales a las del identificador.
    
    \newpage
    
    %Expresion
    \begin{figure}[!htbp]
            \centering
            \includegraphics[width=\textwidth]{chapters/chapter3/figures/diagramas compis-expresion.drawio(1).png}
            \caption{Diagrama Expresión}
            \label{fig:my_label}
    \end{figure}
    \FloatBarrier
    
    \item Si la cima de la pila de operandos es igual a \&\& o $\parallel$, hacer pop a la pila de operandos y generar el cuádruplo de expresión correspondiente a la operador detectado.
    \item Meter $\parallel$ a la pila de operandos
    \item Meter \&\& a la pila de operandos
    
    \newpage
    
    %ExpresionR
    
    \begin{figure}[!htbp]
            \centering
            \includegraphics[width=\textwidth]{chapters/chapter3/figures/diagramas compis-expresionr.drawio.png}
            \caption{Diagrama ExpresiónR}
            \label{fig:my_label}
    \end{figure}
    \FloatBarrier
    
    \item Si la cima de la pila de operandos es igual a <, >, !=, <=, >= ó ==, hacer pop a la pila de operandos y generar el cuádruplo de expresión correspondiente a la operador detectado.
    \item Meter < a la pila de operandos
    \item Meter > a la pila de operandos
    \item Meter != a la pila de operandos
    \item Meter == a la pila de operandos
    \item Meter >= a la pila de operandos
    \item Meter <= a la pila de operandos
    \newpage
    
    %EXP
    \begin{figure}[!htbp]
            \centering
            \includegraphics[width=\textwidth]{chapters/chapter3/figures/diagramas compis-exp.drawio(1).png}
            \caption{Diagrama EXP}
            \label{fig:my_label}
    \end{figure}
    \FloatBarrier
    
    \item Si la cima de la pila de operandos es igual a + ó -, hacer pop a la pila de operandos y generar el cuádruplo de expresión correspondiente a la operador detectado.
    \item Meter + a la pila de operandos
    \item Meter - a la pila de operandos
    \newpage
    
    %Termino
    \begin{figure}[!htbp]
            \centering
            \includegraphics[width=\textwidth]{chapters/chapter3/figures/diagramas compis-termino.drawio(1).png}
            \caption{Diagrama Termino}
            \label{fig:my_label}
    \end{figure}
    \FloatBarrier
    \item Si la cima de la pila de operandos es igual a * ó /, hacer pop a la pila de operandos y generar el cuádruplo de expresión correspondiente a la operador detectado.
    \item Meter * a la pila de operandos
    \item Meter / a la pila de operandos
    \newpage
    
    %Factor
    \begin{figure}[!htbp]
            \centering
            \includegraphics[width=\textwidth]{chapters/chapter3/figures/diagramas compis-factor.drawio(1).png}
            \caption{Diagrama Factor}
            \label{fig:my_label}
    \end{figure}
    \FloatBarrier
    \item Hacer pop a la pila de operador y hacer el cuádruplo de expresión unaria del operador obtenido de la pila
    
    
    \newpage
    
    %VarCte
    \begin{figure}[!htbp]
            \centering
            \includegraphics[width=\textwidth]{chapters/chapter3/figures/diagramas compis-variable_constante.drawio(1).png}
            \caption{Diagrama Variable\_Constante}
            \label{fig:my_label}
    \end{figure}
    \FloatBarrier
    \item Checar si la constante existe en la tabla de constantes, si no existe se agrega a la tabla de constantes con la dirección virtual apropiada. Agregar la dirección virtual de la constante a la pila de operandos
    \item Checar si el id existe en el scope local o global. Si existe agregar su dirección virtual a la pila de operandos.
    \item Checa si la llamada a función es de retorno. Si es de retorno crear una variable temporal con el valor actual de la variable global de la función y agregar el temporal a la pila de operandos
    
    
    \newpage
    
    %Llamada Arreglo
    \begin{figure}[!htbp]
            \centering
            \includegraphics[width=\textwidth]{chapters/chapter3/figures/diagramas compis-llamadaarreglo.drawio.png}
            \caption{Diagrama Llamada Arreglo}
            \label{fig:my_label}
    \end{figure}
    \FloatBarrier
    \item Se revisa si el identificador es un arreglo. 
    \item Se hace pop a la pila de operandos y se revisa que el elemento recibido sea entero. Si es entero se agrega a la fila de subíndices.
    \item Se hace pop a la pila de operandos y se revisa que el elemento recibido sea entero. Si es entero se agrega a la fila de subíndices.
    \item Se hace pop a la pila de operandos y se revisa que el elemento recibido sea entero. Si es entero se agrega a la fila de subíndices.
    \item Se checa que la cantidad de subíndices leídos coincida con las dimensiones del arreglo, de ser así itera por los subíndices generando los cuádruplos de verificación y las sumas apropiadas para calcular la dirección virtual del arreglo accesado.
    
    \newpage
    % Llamada función
    \begin{figure}[!htbp]
            \centering
            \includegraphics[width=\textwidth]{chapters/chapter3/figures/diagramas compis-llamarfuncion.drawio.png}
            \caption{Diagrama Llamada Función}
            \label{fig:my_label}
    \end{figure}
    \FloatBarrier
    
    \item Checar si es una función especial. Si es función especial, crear una variable global con el nombre y tipo de la función y la dirección de memoria apropiada.
    \item Obtener la información de los parámetros del directorio de funciones y generar el cuádruplo de era.
    \item Hacer pop a la pila de operandos y meter el elemento obtenido a una fila de parámetros.
    \item Comparar si los elementos de la fila de parámetros coinciden con los parámetros de la función. Si los elementos coinciden generar los cuádruplos de parameter apropiados para cada elemento de la fila.
    
    \newpage
    % Arreglo Textual
    \begin{figure}[!htbp]
            \centering
            \includegraphics[width=\textwidth]{chapters/chapter3/figures/diagramas compis-arreglo_textual.drawio.png}
            \caption{Diagrama Arreglo\_Textual}
            \label{fig:my_label}
    \end{figure}
    \FloatBarrier
    
    \item Se checa el tipo de la expresión realizada y se mete a un stack auxiliar de tipos y se checa la congruencia de tipo con los demás leídos (si hay).
    \item Se checa la congruencia de la dimensión del elemento leído, si las dimensiones son diferentes se levanta un error. Igual se guarda el tamaño de las dimensiones leídas.
    \item Se calcula el tamaño del arreglo/cubo/matriz  
    
    

\end{enumerate}
\newpage
\FloatBarrier
%---------------------------------------------------------------------------------------
\section{Descripción de Administración de Memoria usado en la compilación}

Para la administración de memoria duarante compilación se usaron princiaplemente los siguientes elementos:

\begin{enumerate}

    \item Directorio de Funciones
    \item Vector de Cuadruplos y su contador
    \item Tabla de Constantes y su espejo
    \item Cubo Semantico
    \item Contadores Globales de Variables Globales y temporales
    \item Máximos de Variables y las direcciones de memoria
    \item Vector de relación linea código fuente a cuadruplo
    \item Variables auxiliares para el manejo de arreglos
    \item Pila de operadores, Pila de operandos y Pila de tipos
    \item Diccionario de funciones especiales
    
\end{enumerate}

Cada uno de estos se va a explicar con más detalle en las siguientes subsecciones.

\subsubsection{Directorio de Funciones y el Directorio de Funciones Especiales}

Para implementar el directorio de funciones se requería una estructura de datos que fuera de acceso rápido y que pudiera contener muchos tipos de variables e información dentro de ella. Afortunadamente en el lenguaje de Python no hay mucha restricción sobre el tipo de datos que se guardan en sus diferentes estructuras. Dado lo que se necesitaba que fuera el directorio de funciones se decidió utilizar la estructura de Diccionario de Python. Este nos permitiría guardar cualquier información necesaria con una llave única, además permite guardar cualquier otra estructura de datos dentro de cada entrada de un diccionario permitiendo una gran flexibilidad de la información que se puede guardar. También los diccionarios son de las estructuras son de rápido acceso para consultas.


\begin{table}[htbp]
    \centering
    %\tiny
    \begin{tabular}{|c|c|c|c|c|c|c|}
        nombre & tipo & address & params & varres & tmpres & vartab \\ \hline
        
         global & vacio & - & - & 1,0,1,1 & - & \tiny Dict de python de var  \\
         pop & vacio & -  & \tiny Dict de python de params & 1,0,3,1 & 14, 1, 2, 7, 0 & \tiny Dict de python de var \\
         principal & vacio & - & \tiny Dict de python de params & 1,0,1,1 & 14, 0, 0, 7, 0 & \tiny Dict de python de var  \\
    \end{tabular}
    \caption{Representación Lógica del directorio de funciones}
    \label{tab:my_label}
\end{table}
\FloatBarrier
Cada entrada de la tabla tiene cinco elementos importantes. El primero es el tipo de la función, este ayuda a identificar si la función tiene retorno o es vacía. La segunda es address. Esta es la dirección virtual de la función si es función de retorno. La tercera es params que es un diccionario de Python que tiene la información del tipo e información de arreglos si el parámetro es un arreglo. El tercero y cuarto elemento son una serie de contadores que representan los recursos locales y temporales respectivamente que tiene una función. El quinto elemento es la tabla de variables que contiene la información de las variables locales de la función.

En el caso de la entrada global, es la única entrada del directorio que no cuenta con información de variables temporales.


Para facilitar la integración de las funciones especiales en compilación también se usó una estructura muy similar al directorio de funciones. De esta manera se pueden usar muchos de los métodos para recorrer el directorio de funciones en el directorio de funciones especiales. Realmente la única diferencia que existe entre los dos es que el directorio de funciones especiales no cuenta con un contador de variables temporales ni una tabla de variables, ya que para ejecutar estas funciones en la máquina virtual solo se necesita saber la información de los parámetros.


\begin{table}[htbp]
    \centering
    %\tiny
    \begin{tabular}{|c|c|c|c|c|c|c|}
        nombre & tipo & address & params & varres \\ \hline
        
         normal & flotante & 1006 & \tiny Dict de python de params  & 2,0,0,0  \\
         modulo & flotante & 1007  & \tiny Dict de python de params & 2,0,0,0  \\
    \end{tabular}
    \caption{Representación Lógica del directorio de funciones especiales}
    \label{tab:my_label}
\end{table}
\FloatBarrier

\FloatBarrier
\subsubsection{Vector de Cuadruplos y su contador}

Para representar los cuádruplos se utilizaron la estructura de tuplas de Python. Pero un programa no cuenta con un solo cuádruplo sino una serie de múltiples cuádruplos ordenados que indican las instrucciones a ejecutar a la máquina virtual. Para lograr esto se creó una lista de tuplas, a la cual se le va agregando como en una fila cada cuádruplo generado. Para tener un mejor control sobre la cantidad de cuádruplos generados también se mantuvo una variable auxiliar global contadora de cuádruplos.

\begin{figure}[htbp]
    \centering
    \begin{lstlisting}[language=Python]
        ('GOTO','','',20)
    \end{lstlisting}
    \caption{Ejemplo de una tupla de Cuádruplo}
    \label{fig:my_label}
\end{figure}

\begin{figure}[htbp]
    \centering
    \begin{lstlisting}[language=Python]
        cuadruplos.append(('GOTOF',res,''))
        global sclines
        sclines.append(p.lineno(1))
        psaltos.append(cuadcount)
        cuadcount += 1
    \end{lstlisting}
    \caption{Ejemplo de generación de un cuádruplo}
    \label{fig:my_label}
\end{figure}
\FloatBarrier

\subsubsection{Tabla de Constantes y su espejo}

Como el directorio de funciones, la tabla de constantes también va a ser una estructura que va a ser consultada muy frecuente mente. Como se había mencionado previamente, el diccionario de Python también es una buena solución para poder guardar y acceder con mucha facilidad información de manera frecuente. Por ello se optó por usar también el diccionario de Python para representar la tabla de constantes.


\begin{table}[htbp]
    \centering
    \begin{tabular}{c|c}
         Constante & Dirección \\ \hline
         25000 & 2 \\
         25001 & 1 \\
         310000 & "Hola" \\
    \end{tabular}
    \caption{Representación Lógica de la tabla de constantes}
    \label{tab:my_label}
\end{table}
\FloatBarrier

\begin{figure}[htbp]
    \centering
    \begin{lstlisting}[language=Python]
        {                                         
            "2": 25000,                            
            "1": 25001,                            
            "3": 25002,                            
            "1001": 25003,                         
            "1003": 25004,                         
            "3000": 25005,                         
            "0": 25006,                            
            "5": 25007,                            
            "\"Factorial de n \\n\"": 31000,       
            "10": 25008,                           
            "\"Factorial Secuencial : \"": 31001,  
            "\"------\\n hola aka =\"": 31002      
        }                                           
    \end{lstlisting}
    \caption{Ejemplo de el diccionario de constantes}
    \label{fig:my_label}
\end{figure}
\FloatBarrier

Como Python no es un lenguaje fuertemente tipado, para generar las llaves del diccionario se transforma a string la constante para poderla encontrar fácilmente. También para reducir el tiempo de ejecución a la hora de crear el archivo obj, se mantiene una copia de la tabla de constantes con el formato correcto para la máquina virtual.
\begin{figure}[htbp]
    \centering
    \begin{lstlisting}[language=Python]
        {
            "25000": 3,
            "25001": 1,
            "25002": 9,
            "25003": 9003,
            "25004": 14001,
            "25005": 9006,
            "25006": 10,
            "25007": 30,
            "25008": 2,
            "25009": 1000000,
            "25010": 20,
            "30000": true,
            "31000": "\"Hola Mundo\"",
            "25011": 0,
            "31002": "\"Iteracion por la matriz\"",
            "31003": "\"Iteracion por el cubo\"",
            "31004": "\"\\t--Dim2\"",
            "31005": "\"\\t\\t---Dim 3\"",
            "31006": "\"Probando arreglos textuales\"",
            "29000": "'a'",
            "25012": 4,
            "25013": 5,
            "25014": 6,
            "29001": "'b'",
            "29002": "'c'",
            "30001": false,
            "25015": 127
        }
    \end{lstlisting}
    \caption{Ejemplo de el diccionario de constantes para la máquina virtual}
    \label{fig:my_label}
\end{figure}

\FloatBarrier

\subsection{Tabla de variables}

Para la tabla de variables también se necesitó una estructura de datos que se rápida de acceder y modificar. Como se mencionó con el directorio de funciones, el diccionario de Python resulto ser una estructura de datos que se adaptaba muy bien a solucionar este problema. Por ello también se utilizó un diccionario de Python para representar la tabla de variables.


\begin{table}[htbp]
    \centering
    \begin{tabular}{c|c|c|c|c}
        nombre & tipo & dirección & dims & dimlen   \\
        a & entero & 1000 & - & - \\
        b & entero & 1001 & 2 & [3,4], [4,0] \\
    \end{tabular}
    \caption{Representación Lógica de la tabla de variables}
    \label{tab:my_label}
\end{table}
\FloatBarrier
El diccionario de la tabla de variables constaba de que cada llave fuera el nombre de la variable y que dentro de esa llave se guardara otro diccionario con la información de la variable. Este diccionario tiene lo que es el tipo de la variable, su dirección de memoria, y si es un arreglo, las dimensiones del arreglo y la información para calcular la indexación.

\subsubsection{Cubo Semántico}

Para el cubo semántico se decidió ir por una estructura que pudiera ser accesada rápidamente y que pudiera dar el resultado de una operación con llaves sencillas. Ya que se trabajó en el lenguaje de Python, se aprovechó de la estructura nativa de diccionarios. Estos pueden ser accesados rápidamente y se le pueden asignar llaves a cada entrada. Para hacer el código legible se decidió estructurar el cubo como un diccionario de diccionarios. De esta manera la llamada al cubo semántico sería accesando al cubo con la primera llave siendo el operador, y las otras dos llaves siendo el operando izquierdo y el operando derecho.


\begin{table}
    \centering
    \caption{Tabla representando el cubo semántico}
    \small
    \begin{tabular}{||c c || c c c c c c c c c c c c c||} 
         
         OpIzq & OpDer & $=$ & $\parallel$ & \&\& & \textless & \textgreater & \textless $=$ & \textgreater $=$ & !$=$ & $==$ & $+$ & $-$ & * & \textbackslash \\ [0.5ex] 
         \hline\hline
         ent & ent & ent & bool & bool & bool & bool & bool & bool & bool & bool & ent & ent & ent & ent \\ 
         \hline
         ent & flot & ent & bool & bool & bool & bool & bool & bool & bool & bool & flot & flot & flot & flot \\ 
         \hline
         ent & char & ent & bool & bool & bool & bool & bool & bool & bool & bool & ent & ent & ent & ent \\ 
         \hline
         ent & bool & ent & bool & bool & bool & bool & bool & bool & bool & bool & ent & ent & ent & ent \\ 
         \hline
         ent & cadena & err & err & err & err & err & err & err & err & err & err & err & err & err \\ 
         \hline
         flot & ent & flot & bool & bool & bool & bool & bool & bool & bool & bool & flot & flot & flot & flot \\ 
         \hline
         flot & flot & flot & bool & bool & bool & bool & bool & bool & bool & bool & flot & flot & flot & flot \\ 
         \hline
         flot & char & flot & bool & bool & bool & bool & bool & bool & bool & bool & flot & flot & flot & flot \\ 
         \hline
         flot & bool & flot & bool & bool & bool & bool & bool & bool & bool & bool & flot & flot & flot & flot \\ 
         \hline
         flot & cadena & err & err & err & err & err & err & err & err & err & err & err & err & err \\ 
         \hline
         char & ent & char & bool & bool & bool & bool & bool & bool & bool & bool & ent & ent & ent & ent \\ 
         \hline
         char & flot & char & bool & bool & bool & bool & bool & bool & bool & bool & flot & flot & flot & flot \\ 
         \hline
         char & char & char & bool & bool & bool & bool & bool & bool & bool & bool & ent & ent & ent & ent \\ 
         \hline
         char & bool & char & bool & bool & bool & bool & bool & bool & bool & bool & ent & ent & ent & ent \\ 
         \hline
         char & cadena & err & err & err & err & err & err & err & err & err & err & err & err & err \\ 
         \hline
         bool & ent & bool & bool & bool & bool & bool & bool & bool & bool & bool & ent & ent & ent & ent \\ 
         \hline
         bool & flot & bool & bool & bool & bool & bool & bool & bool & bool & bool & flot & flot & flot & flot \\ 
         \hline
         bool & char & bool & bool & bool & bool & bool & bool & bool & bool & bool & ent & ent & ent & ent \\ 
         \hline
         bool & bool & bool & bool & bool & bool & bool & bool & bool & bool & bool & ent & ent & ent & ent \\ 
         \hline
         bool & cadena & err & err & err & err & err & err & err & err & err & err & err & err & err \\ 
         \hline
         cadena & ent & err & err & err & err & err & err & err & err & err & err & err & err & err \\ 
         \hline
         cadena & flot & err & err & err & err & err & err & err & err & err & err & err & err & err \\ 
         \hline
         cadena & char & err & err & err & err & err & err & err & err & err & err & err & err & err \\ 
         \hline
         cadena & bool & err & err & err & err & err & err & err & err & err & err & err & err & err \\ 
         \hline
         cadena & err & err & err & err & err & err & err & err & err & err & err & err & err & err \\ 
        
         
        
        \end{tabular}
        \begin{itemize}
            \item ent : Entero
            \item flot : Flotante
            \item err : Error
        \end{itemize}
    
\end{table}
\FloatBarrier

\begin{figure}[htbp]
    \centering
    \begin{lstlisting}[language=Python]
        rettype = ptipo.pop()
        if (cubosem['='][functipo][rettype] != 'error'):
            dprint('Cubo dice: ',cubosem['='][functipo][rettype])
            retop = pilaoperand.pop()
            ......
    \end{lstlisting}
    \caption{Ejemplo del uso de el Cubo Semántico}
    \label{fig:my_label}
\end{figure}
\FloatBarrier
\begin{figure}[htbp]
    \centering
    \begin{lstlisting}[language=Python]
        cubosem = {
        '=':{
            'entero':{
                        'entero':'entero',
                        'flotante': 'entero',
                        'char':'entero',
                        'bool': 'entero',
                        'cadena':'error'
                    },
            'flotante':{
                        'entero':'flotante',
                        'flotante': 'flotante',
                        'char':'flotante',
                        'bool': 'flotante',
                        'cadena':'error'
                    },
            'char':{
                        'entero':'char',
                        'flotante': 'char',
                        'char':'char',
                        'bool': 'char',
                        'cadena':'error'
                    },
            'bool':{
                        'entero':'bool',
                        'flotante': 'bool',
                        'char':'bool',
                        'bool': 'bool',
                        'cadena':'error'
                    },
            'cadena':{
                        'entero':'error',
                        'flotante': 'error',
                        'char':'error',
                        'bool': 'error',
                        'cadena':'err'
                    },

            },
        .....
    \end{lstlisting}
    \caption{Muestra de la primera entrada del cubo semántico}
    \label{fig:my_label}
\end{figure}
\FloatBarrier




\subsubsection{Contadores Globales de Variables Globales y temporales, Máximos de Variables y las Direcciones de Memoria}

Para mantener un registro del uso de recursos dentro del programa durante el proceso de compilación se crearon variables globales que funcionaban como contadores sobre cada tipo de variable. Cada vez que se generaba una nueva variable se aumentaba el contador y se verificaba que no se pasara de los máximos del tipo permitidos. Estos contadores se dejaron como variables globales porque había múltiples acciones semánticas que requerían consultar o actualizar sus valores. Para mantener legibilidad y por simplicidad se optó por dejarlo como variables globales.

\begin{table}[htbp]
    \centering
    \begin{tabular}{c|c|c|c|c}
        \multicolumn{5}{c}{Globales} \\\hline
         Enteros & Flotantes & Caracteres & Booleanos & Apuntadores  \\
          1 & 2 & 0 & 3 & 0 \\\hline\hline
          
          \multicolumn{5}{c}{Temporales} \\
         Enteros & Flotantes & Caracteres & Booleanos & Apuntadores  \\
          8 & 2 & 0 & 6 & 0 
    \end{tabular}
    \caption{Representación Lógica de los Contadores Globales}
    \label{tab:my_label}
\end{table}
\FloatBarrier
Similar que con los contadores globales también se dejó los máximos de variables y de direcciones de memoria como variables globales para mantener legibilidad y simplicidad en el código. Como muchas de las acciones semánticas también interactuar con ellas se mantienen globales.

\subsubsection{Vector de relación línea código fuente a cuádruplo}

Para desarrollar mensajes de error más detallados en la máquina virtual, se necesitaba saber en qué línea de código se causó el error. Para resolver este problema se creó una lista que aprovecha el rastreador de tokens del lexer de \emph{ply} para guardar el número de la línea del código fuente en relación con un cuádruplo. Esta lista va creciendo a la par que la lista de cuádruplos y es pasada como información a la máquina virtual para que pueda usarla en sus mensajes de error.


\begin{figure}[htbp]
    \centering
    \begin{lstlisting}
    // La lista de cuadruplos
        Cuadruplos = [('GOTO','MAIN','',''), 
        ("*", 25006, 25007, 17000), 
        ("/", 17000, 25007, 17001), 
        ("*", 17001, 25008, 17002), 
        ......]
    // Lista de la linea de codigo fuente que la creo
        sclines = [1, 9, 9, 9, 9, ......]
    \end{lstlisting}
    \caption{Ejemplo de como se ve la relación de cuádruplos contra líneas de código fuente}
    \label{fig:my_label}
\end{figure}
\FloatBarrier
\subsubsection{Variables auxiliares para el manejo de arreglos}

Para el cálculo de la información de la indexación de arreglos se optó por dejar las variables como variables globales. Como no son muchas las variables para el cálculo de la fórmula de indexación, ya que este lenguaje se limita hasta cubos nada más, no se vio necesario crear una estructura extra para manejar su información. De igual manera, como son múltiples acciones semánticas que utilizan estas variables por conveniencia se dejaron como variables globales.


\begin{table}[htbp]
    \centering
    \begin{tabular}{|c|c|}
       R  &  1\\
      at1 & 2 \\
      at2 & 3 \\
      at3 & 0 \\
    \end{tabular}
    \caption{Representación Lógica de las Varables Auxiliares de arreglos}
    \label{tab:my_label}
\end{table}

\FloatBarrier
\subsubsection{Pila de operadores, Pila de operandos y Pila de tipos}

Para el manejo de la lógica de expresiones como visto en clase se necesitaba utilizar una estructura stack para el manejo de las acciones semánticas. Python no cuenta con una estructura stack, pero cuenta con métodos que simulan a un stack en su estructura de lista. Por ello se manejó la Pila de Operadores, Pila de Operandos y la Pila de tipos como listas las cuales solo se podían interactuar con usando los métodos de \emph{pop} y \emph{append} que tienen las listas.

\begin{table}[htbp]
    \centering
    \begin{tabular}{c|c c c c}
         \hline
        Pila de Operadores & + & * & & ....   \\\hline \hline
        Pila de Operandos &  A & 1000 & x\_inicial & ....\\\hline \hline
        Pila de Tipos & entero & entero & entero & .... \\\hline \hline
    \end{tabular}
    \caption{Representación Lógica de las Pilas}
    \label{tab:my_label}
\end{table}


%\part{Descripción de la máquina virtual}
\chapter{Descripción de la máquina virtual}

En esta parte se describe con más detalle lo que se utilizo para la máquina virtual y su manejo de memoria

\section{Equipo de cómputo, lenguaje y utilerías especiales usadas}

Una PC con Windows 10, se utilizó el lenguaje de programación Python 3.10 con apoyo de las librerías de \emph{PLY, Numpy, re, json, SciPy y Matplotlib}.

La máquina virtual utiliza las librerías de emph{JSON, SciPy, NumPy y Matplotlib}. Utiliza la librería de \emph{JSON} para leer el archivo generado por el compilador y con toda la información del archivo va ejecutando el código intermedio generado por el compilador. Cuando la máquina lee una función especial utiliza las librerías de \emph{SciPy, NumPy y Matplotlib} para ejecutar dichas funciones.


\section{Descripción del proceso de Administración de Memoria en ejecución}
Para lograr ejecutar de manera correcta el código compilado es importante poder representar y manejar el espacio de memoria en la máquina virtual, de manera en la que se pueda convertir las direcciones virtuales asignadas por el compilador a espacios de memoria en la máquina.

\subsection{La Clase Memoria}

Para manejar la memoria en la máquina virtual se desarrollo la clase Memory, la cual representa un conjunto de espacios de memoria para enteros, flotantes, caracteres, booleanos y apuntadores como cinco listas. Esta clase recibe un número representando la cantidad de espacios para cada tipo de variable en su constructor y se genera una lista con el tamaño apropiado.

%Diagrama de clase de Memory
\begin{figure}[htbp]
    \centering
    \includegraphics[scale=0.6]{chapters/chapter4/figures/Class_Memeory.png}
    \caption{Diagrama de Clase de Memory}
    \label{fig:my_label}
\end{figure}
\FloatBarrier
De esta manera se puede mapear la memoria de la máquina virtual, las listas que contienen la información, a las direcciones virtuales del compilador realizando la siguiente formula.
$$Espacio\_en\_el\_arreglo = Dir\_Virtual - Dir\_Virtual\_Base\_del\_tipo$$
Con esta formula se puede sacar la posición en el arreglo se encuentra el valor de la variable.


\subsection{Manejo de memoria global, local y temporal}

Ahora para representar el espacio de memoria en la máquina virtual se necesita primero plantear el como se debe de manejar los scopes de las variables. Para manejar las variables globales basta con crear una instancia de Memory para representarlas, pero se empieza a complicar la situación cuando se intenta representar el scope de la función principal o de cualquier otra función. Esto se debe a que Memory solo cuenta con un conjunto de arreglos, pero estos espacios de memoria cuentan no solo con variables locales sino que también con variables temporales. Para lidiar con este problema el espacio de memoria de una función se representa por un vector con dos instancias de Memory. La primera representando las variables locales y la segunda representando las variables temporales.
% imagen representando esto

\begin{figure}
    \centering
    \begin{lstlisting}[language=Python]
        globalMem = Memory(1,6,0,3)
        # El primero representa el local y el segundo el temporal
        principalMem = [Memory(1,6,0,3), Memory(7,6,0,8)] 
    \end{lstlisting}
    \caption{Representación en código de las memorias}
    \label{fig:my_label}
\end{figure}
\FloatBarrier

Con este vector ahora ya se puede manejar todos los posibles scopes de la memoria.


\subsection{El stack de memorias}

Habiendo resuelto el problema de las variables temporales y locales, surge un nuevo problema. ¿Cómo se sabe en qué memoria se está trabajando actualmente? Para resolver esto se implemento una pila, la cual maneja el contexto actual con el que se esta trabajando. La idea de usar una pila viene a la naturaleza del uso de espacios de memoria cuando se llama a una función. 
Lógicamente si seguimos el algoritmo que generan los cuádruplos, la máquina virtual va a parar la ejecución del código actual y va a saltar a realizar el código de la función. Después de acabar la función la máquina debe de regresar a donde estaba y continuar el código. Esto se podría resolver con una variable auxiliar, pero surge un problema si se utiliza una variable auxiliar cuando una función llama a otra función. 
% ejemplo con listings

\begin{figure}[htbp]
    \centering
    \begin{lstlisting}[language = Python]
        # Se llama funcion 1
        auxMem = principalMem
        .........
        # Se llama funcion 2
        auxMem = funcMem
        # Ya se perido la memoria de principal 
    \end{lstlisting}
    \caption{Representación de el uso de una variable auxiliar}
    \label{fig:my_label}
\end{figure}
\FloatBarrier
Si solo se usa una variable auxiliar la nueva función va a causar que la memoria de la función que se anda ejecutando sobre escriba a la memoria que la llamo. Pero este problema se puede resolver con una pila. Si una función llama a otra función en su ejecución lo que se puede hacer es ir guardando las memorias en la pila y cuando se dejen de utilizar se remueven del tope de la pila y se continúa utilizando el siguiente elemento en la pila. De esta manera se resuelve el problema de cuando una función llama a otra función.
\begin{figure}[htbp]
    \centering
    \begin{lstlisting}[language = Python]
        # Se llama funcion 1
        auxMem.append(principalMem)
        .........
        # Se llama funcion 2
        auxMem.append(funcMem)
        # Ahora solo se tiene que hacer pop al stack para obtener el contexto pasado
    \end{lstlisting}
    \caption{Representación de el uso de una pila para el manejo de memoria}
    \label{fig:my_label}
\end{figure}
\FloatBarrier
Esta solución no solo aplica para las funciones que son dependientes a otras, sino que también soluciona el problema de el manejo de memorias en funciones recursivas.

% ejemplo con tabla
\begin{table}[htbp]
    \centering
    \begin{tabular}{c|c c c c}
       
        Pila de Memorias & MemPrincipal & MemFuncRec & MemFuncRec & ..... \\
         
    \end{tabular}
    \caption{Representación de la pila en llamadas recursivas}
    \label{tab:my_label}
\end{table}

\section{Constantes}

Para el manejo de constantes en la memoria de la máquina virtual no fue necesario crear un espacio de memoria con la clase Memory. Para reducir la cantidad de instancias generadas se opto por pasar la tabla de constantes de compilación al archivo obj que resulta del proceso. De esta manera se puede recuperar esta información cuando se lee el archivo.  Esta información es guardada en un diccionario que puede ser accesado con la dirección virtual como la llave y nos ahorra el calculo de obtener la posición que debería de tener en memoria.

\part{Pruebas del funcionamiento del lenguaje}
\chapter{Pruebas del funcionamiento del lenguaje }

En esta parte se demuestra código escrito en el lenguaje, y como el compilador y la máquina virtual interactuan con este código fuente.

\section{Programa Factorial}
En este programa se implementaron dos formas para calcular la factorial de un número. Una usando una función recursiva y la otra usando un método secuencial.
A continuación, se presenta el código fuente:
\begin{figure}[htbp]
    \centering
    \begin{lstlisting}
        programa factorial { 
	
    	entero n;
    	entero r[ 2 ];
    	entero l[ 3 ][ 2 ];
    	flotante g[1][1][1];
    	
    	funcion entero factorial (entero num){}{
    		si(num > 0){
    			regresar num * $factorial(num-1);
    		}sino{
    			regresar 1;
    		}
    	}
    
    	
    
    	principal funcion
    	{
    		entero aka,aux,aux2;
    		
    
    	}
    	{
    		n = $factorial(5);
    		imprimir("Factorial de n \n");
    		imprimir(n);
    		aka = 10-n;
    		10 / (-aka) ;
    		aka = 5;
    		aux2 = aka;
    		aux = 1;
    		mientras(aux <= aux2){
    			si (aux2 - aux > 0 ){ 
    				aka = aka * (aux2-aux);
    			}sino{
    				aka = aka * 1;
    			}
    			aux = aux + 1;
    		}
    		imprimir("Factorial Secuencial : ",aka);
    		
    	} 
    
    }

    \end{lstlisting}
    \caption{Código fuente del programa factorial}
    \label{fig:my_label}
\end{figure}
\FloatBarrier

A continuación se muestra su ejecución en consola:
\begin{figure}[htbp]
    \centering
    \includegraphics[scale=0.5]{chapters/chapter5/figures/FactorialCorriendo.JPG}
    \caption{El programa factorial compilado y ejecutado}
    \label{fig:my_label}
\end{figure}
\FloatBarrier

\section{Programa Fibonacci}

En este programa se implementaron dos formas para calcular el número n de la serie de Fibonacci. Una usando una función recursiva y la otra usando un método secuencial.
A continuación, se presenta el código fuente:
\begin{figure}[htbp]
    \centering
    \begin{lstlisting}
        programa fibbo { 
    	funcion entero fibbo(entero n){}{
    		si(n == 0){
    			regresar 0;
    		}
    		si(n == 1){
    			regresar 1;
    		}
    		regresar $fibbo(n-1) + $fibbo(n-2);
    	}
    	principal funcion {
    		entero pri,seg,ter,obj,aux;
    	} {	
    		obj = 8;
    		imprimir("fibbo ",$fibbo(obj));
    
    		pri = 0;
    		seg = 1;
    		aux = 1;
    		mientras( aux < obj){
    			ter = pri + seg;
    			pri = seg;
    			seg = ter;
    			aux = aux + 1;
    		}
    		imprimir("Fibbo secuencial : ",ter);
    
    	} 
}

    \end{lstlisting}
    \caption{Código fuente del programa fibbo}
    \label{fig:my_label}
\end{figure}
\FloatBarrier

A continuación se muestra su ejecución en consola:
\begin{figure}[htbp]
    \centering
    \includegraphics[scale=0.5]{chapters/chapter5/figures/fibbocorriendo.JPG}
    \caption{El programa fibbo compilado y ejecutado}
    \label{fig:my_label}
\end{figure}
\FloatBarrier


\section{Programa Busqueda y organización de arreglo}

En este programa se implementaron una función de búsqueda en un arreglo y un Bubble Sort para organizar un arreglo.
A continuación, se presenta el código fuente:
\begin{figure}[htbp]
    \centering
    \tiny
    \begin{lstlisting}
        programa arraysearchnsort{
            entero a[10];
            funcion vacio imprimirArr(entero a[10]){
                entero x;
            }{
                por(x = 0; x < 10; x = x+1;){
                    imprimir("a[",x,"] = ",a[x]);
                }
            }
        
            funcion vacio encontrar(entero x, entero a[10]){
                entero i,aux,aux2;
                bool loEncontre;
            }{
                imprimir("\n------------------\nBuscando ",x," en el arreglo :");
                $imprimirArr(a);
                imprimir("\n\n");
                aux = -1;
                loEncontre = falso;
                por(i = 0;i < 10;i = i +1;){
                    si(a[i] == x){
                        
                        aux2 = i;
                        i = 10+1;
                        loEncontre = verdadero;
                    }
                }
                si (loEncontre){
                    imprimir("Encontro ",x," en a[",aux2,"]");
                }sino{
                    imprimir("No se encontro el numero");
                }
            }
            principal funcion {
                entero i, j, aux;
            } {
                
                a = [1,5,4,3,2,-7,9,8,6,10];
                $encontrar(4,a);
                $encontrar(-1,a);
                imprimir("Desordenado:\n");
                $imprimirArr(a);
                por(i = 0; i < 10; i = i+1;){
                    por(j = 0; j < 10 -i-1; j = j+1;){
                        si(a[j] > a[j+1]){
                            aux = a[j];
                            a[j] = a[j+1];
                            a[j+1] = aux;
                        }
                    }
                }
                imprimir("\n\nOrdenado:\n");
                $imprimirArr(a);
                imprimir();
            }
        }
    \end{lstlisting}
    \caption{Código fuente del programa arraysearchnsort}
    \label{fig:my_label}
\end{figure}
\FloatBarrier
A continuación se muestra su ejecución en consola:
\begin{figure}[htbp]
    \centering
    \includegraphics[scale=0.5]{chapters/chapter5/figures/arraycorriendo1.JPG}
    \caption{El Búsqueda y Sort compilado y ejecutado parte 1}
    \label{fig:my_label}
\end{figure}
\FloatBarrier
\begin{figure}[htbp]
    \centering
    \includegraphics[scale=0.5]{chapters/chapter5/figures/arraycorriendo2.JPG}
    \caption{El Búsqueda y Sort compilado y ejecutado parte 2}
    \label{fig:my_label}
\end{figure}
\FloatBarrier

\section{Programa Multiplicación de Matrices}
En este programa se implemento una función de multiplicación de matrices.
A continuación, se presenta el código fuente:
\begin{figure}[htbp]
    \centering
    \tiny
    \begin{lstlisting}
        programa matmul{
            entero matA[2][3];
            entero matB[3][4];
            entero r1, comun ,c2;
            funcion vacio imprimirMatA(entero a[2][3]){
                entero x,y;
            }{
                por(x = 0; x < 2; x = x+1;){
                    por(y = 0; y < 3; y = y+1;){
                        imprimir("mat[",x ,"][", y,"] = ",a[x][y]);
                    }
                }
                
            }
            funcion vacio imprimirMatB(entero a[3][4]){
                entero x,y;
            }{
                por(x = 0; x < 3; x = x+1;){
                    por(y = 0; y < 4; y = y+1;){
                        imprimir("mat[",x ,"][", y,"] = ",a[x][y]);
                    }
                }
                
            }
        
            funcion vacio imprimirMat(entero a[2][4]){
                entero x,y;
            }{
                por(x = 0; x < 2; x = x+1;){
                    por(y = 0; y < 4; y = y+1;){
                        imprimir("mat[",x ,"][", y,"] = ",a[x][y]);
                    }
                }
                
            }
        
            principal funcion{
                entero matR[2][4];
                entero i,j,k;
            }{
                r1 = 2;
                comun = 3;
                c2 = 4;
                matA = [[1,2,3],[4,5,6]];
                imprimir("Matriz A :");
                $imprimirMatA(matA);
                matB = [[1,2,3,4],[5,6,7,8],[9,10,11,12]];
                imprimir("Matriz B :");
                $imprimirMatB(matB);
                
                por (i = 0; i < r1; i = i+1;) { //Iterar sobre los renglones de la primera matriz
                    por(j = 0; j < c2; j = j +1;){// Iterar sobre las columnas de la segunda matriz
                        matR[i][j] = 0;
                        por(k = 0; k < comun; k = k+1;){ 
                        // Recorer la col/reg para  hacer la multiplicacion
                            matR[i][j] = matR[i][j] + matA[i][k] * matB[k][j]; 
                            //matR[i][j] + matA[R_i][k] * matB[k][C_j]
                        }
                         
                    }
                }
        
                imprimir("Matriz R :");
                $imprimirMat(matR);
            }
        }
    \end{lstlisting}
    \caption{Código fuente del programa matmul}
    \label{fig:my_label}
\end{figure}
\FloatBarrier
A continuación se muestra su ejecución en consola:
\begin{figure}[htbp]
    \centering
    \includegraphics[scale=0.5]{chapters/chapter5/figures/matmulcorriendo.JPG}
    \caption{El matmul compilado y ejecutado}
    \label{fig:my_label}
\end{figure}
\FloatBarrier
\section{Programa Uso de Funciones de estadística}

En este programa se implementaron unas pruebas de Wilcoxon. Se crearon tres arreglos que simulaban muestras las cuales se popularon con números aleatorios pertenecientes a una distribución y se desplego la información básica de cada arreglo. Al final se hicieron unas pruebas  de wilcoxon para probar cual de las muestras se acomodaba mejor.

A continuación, se presenta el código fuente:
\begin{figure}[htbp]
    \centering
    \tiny
    \begin{lstlisting}
        programa pruebasDeWilcoxon{
                // La declaracion de variables
                flotante muestra1[100];
                flotante muestra2[100];
                // Declaracion de funciones
                funcion vacio imprimirDatos( flotante arr[100] ){
                    flotante prom, mod, med;
                }{
                    prom = $media(arr);
                    mod = $moda(arr);
                    med = $mediana(arr);
                    imprimir(" La media de esta distribucion es : ", prom);
                    imprimir(" La mediana de esta distribucion es : ", med);
                    imprimir(" La moda de esta distribucion es : ", mod);
                    imprimir(" El cuartil del 50 de los datos : ", $percentil(arr,50));
                }
                //Funcion principal
                principal funcion {
                    flotante muestra3[100];
                    entero i;
                    flotante p;
                }{
                    por(i = 0; i < 100; i = i+1;){
                        muestra1[i] = $normal(3.5,0.6);
                        muestra2[i] = $dexponencial(4);
                        muestra3[i] = $poisson(2);
                    }
            
                    imprimir("Muestra 1 : Normal");
                    $imprimirDatos(muestra1);
                    imprimir("Muestra 2 : Exponencial");
                    $imprimirDatos(muestra2);
                    imprimir("Muestra 3 : Poisson");
                    $imprimirDatos(muestra3);
            
                    //Pasar una prueba de wilcoxon p > 0.5
                    p = $wilcoxonComp(muestra1,muestra2);
                    imprimir("\n\nMuestra 1 v Muestra 2");
                    si(p > 0.5){
                        imprimir("Muestra 1 es la que mejor representa los datos");
                    }sino{
                        imprimir("Muestra 2 es la que mejor representa los datos");
                    }
            
                    p = $wilcoxonComp(muestra1,muestra3);
                    imprimir("\n\nMuestra 1 v Muestra 3");
                    si(p > 0.5){
                        imprimir("Muestra 1 es la que mejor representa los datos");
                    }sino{
                        imprimir("Muestra 3 es la que mejor representa los datos");
                    }
            
                    p = $wilcoxonComp(muestra3,muestra2);
                    imprimir("\n\nMuestra 3 v Muestra 2");
                    si(p > 0.5){
                        imprimir("Muestra 3 es la que mejor representa los datos");
                    }sino{
                        imprimir("Muestra 2 es la que mejor representa los datos");
                    }
            
                }
            
        }
    \end{lstlisting}
    \caption{Código fuente del programa}
    \label{fig:my_label}
\end{figure}
\FloatBarrier
A continuación se muestra su ejecución en consola:
\begin{figure}[htbp]
    \centering
    \includegraphics[scale=0.4]{chapters/chapter5/figures/wilcoxCorriendo.JPG}
    \caption{El programa de pruebasDeWilcoxon compilado y ejecutado}
    \label{fig:my_label}
\end{figure}

\part{Manual de Usuario}
\section{Introducción}

IntroProg es un lenguaje de programación diseñado para principiantes que quieran aprender a programar con lenguajes estilo C/C++ experimentando con funciones estadísticas.

\section{Herramientas que utiliza}
IntroProg se logró hacer con la ayuda de las siguientes herramientas:
\begin{itemize}
\item Python
\item Ply
\item SciPy
\item NumPy
\item MatPlotLib
\end{itemize}
\section{Requerimientos}

Para poder utilizar el lenguaje lo único que se requiere es tener Python instalado con las librerías que se mencionaron en las herramientas.

\section{Como correr un programa IntroProg}

Para correr tu programa IntroProg debes de primero compilarlo con el archivo de IntroProg.py que se encuentra dentro de la carpeta lexnsyn
\begin{lstlisting}
py IntroProg.py camino/a/tu/archivo
\end{lstlisting}

Esto generara un archivo obj el cual la máquina virtual que está en la carpeta de virtualmachine va a usar para ejecutar tu código

\begin{lstlisting}
py itvm.py camino/a/tu/archivo/obj
\end{lstlisting}

\section{Como utilizar el lenguaje}

Para poder utilizar el lenguaje lo primero que debes de hacer es crear un archivo itp. Este archivo deberá contener tu código fuente que vayas creando.
Los archivos itp siguen la siguiente estructura:
\begin{lstlisting}

programa pelos{
    //Primero deben de ir la declaracion de variables globales
    //Luego deben de ir las declaracion de tus funciones
    //Luego va la función principal
    principal funcion{}{
        imprimir("¡¡Hola Mundo!!");
    }

}

\end{lstlisting}

Los tipos que utiliza IntroProg para sus variables son los siguientes:
\begin{itemize}
    \item Enteros
    \item Flotantes
    \item Caracteres (char)
    \item Booleanos (bool)
\end{itemize}

Para la declaración de una variable basta con escribir el tipo de la variable seguido por el nombre de la variable. Este nombre puede ser cualquier combinación de letras, números y guiones bajos (\_) que empiezen con una letra.

\begin{lstlisting}
    entero num;
    flotante num2;
    char letra;
    bool booleano1;
\end{lstlisting}

\section{Operaciones}

Para realizar operaciones con IntroProg basta con escribirlas como si fueran operaciones matemáticas.
\begin{lstlisting}
    // suma
    num = 1+1;
    // resta
    num = num -1;
    // multiplicacion
    num2 = num \item 20;
    // Division
    num2 = num /num2;
    // Tambien se pueden usar parentesis
    num2 = (1+1)\itemnum - num2 \item num2;
    // Taqmbién puedes hacer funciones relacionales y lógicas
    booleano1 = num2 > 1 || (num2 == num && num <> 1); 
\end{lstlisting}

\section{Funciones}

 Para construir funciones en IntroProg debes de seguir la siguiente estructura:

\begin{lstlisting}
  funcion tipo nombre(parametros){
      //Declaraciones
      // Se puede dejar vacio si no se usan variables locales
    }{
        //Tu asmbroso codigo
    }
\end{lstlisting}

Recuerda que puedes tener funciones que no tengan parámetros y no regresen nada

\begin{lstlisting}
    funcion vacia hola(){}{
        imprimir("Hola!!");
    }
\end{lstlisting}

O funciones que puedan hacer operaciones y que te regresen un valor
\begin{lstlisting}
    funcion entera sumaMagica(entero a, entero b){
        entero ingredienteSecreto;
    }{
        ingredienteSecreto = 1000;
        regresar a + b\itemingredienteSecreto;
    }
\end{lstlisting}
Y para utilizar tus funciones solo basta con escribir un \$ seguido por el nombre de tu función y sus parametros entre paréntesis

\begin{lstlisting}
    \$hola();
    num = \$sumaMagica(1,2);
\end{lstlisting}

\section{Funciones especiales}

IntroProg también te provee una lista de funciones que puedes utilizar para tu código
\begin{itemize}
    \item  \$leer() : Regresa la lectura capturada de consola
    \item  \$modulo(flotante a, flotante b): Regresa el resultado de a\%b
    \item  \$suma(flotante a[]): Suma todos los elementos de un arreglo y regresa sus resultados
    \item  \$raiz(flotante a): Regresa el flotante resultante de la raíz de a
    \item  \$exp(flotante a): Regresa la exponencial de e\^a
    \item  \$elevar(flotante a, flotante b): Regresa el resultado de a\^b
    \item  \$techo(flotante a): Regresa un flotante a redondeado para arriba
    \item  \$piso(flotante a): Regresa un flotante a redondeado para abajo
    \item  \$cos(flotante a): Regresa el coseno de a
    \item  \$sen(flotante a): Regresa el seno de a
    \item  \$tan(flotante a): Regresa la tangente de a
    \item  \$cotan(flotante a): Regresa la cotangente de a
    \item  \$sec(flotante a): Regresa la secante de a
    \item  \$cosec(flotante a): Regresa la cosecante de a
    \item  \$log(flotante a): Regresa el logaritmo natural de a
    \item  \$minimo(flotante a[]): Regresa el valor más chico en el vector a
    \item  \$maximo(flotante a[]): Regresa el valor máximo de a
    \item  \$redondear(flotante a): Regresa un flotante a redondeado
    \item  \$productoPunto(flotante a[], flotante b[]): Regresa el producto punto entre los vectores de entrada a y b
    \item  \$media(flotante a[]): Regresa la media de a
    \item  \$mediana(flotante a[]): regresa la mediana de a
    \item  \$moda(flotante a[]): Regresa el elemento con la moda más alta de a
    \item  \$varianza(a): Regresa la varianza de a
    \item  \$percentil(flotante a[], flotante q): Regresa el valor en el que se encuentran q% de los valores en a
    \item  \$aleatorio(flotante min, flotante max): Regresa un número flotante aleatorio entre los rangos de argumentos mínimos y máximos
    \item  \$wilcoxon(flotante x[]): Realiza la prueba de Wilcoxon en la serie de datos en x
    \item  \$wilcoxonComp(flotante x[], flotante y[]):Realiza la prueba de Wilcoxon se realiza la prueba sobre los datos x y y
    \item  \$regresionSimple(flotante x[], flotante y[], flotante xi): Dado un set de x y y se usará regresión lineal simple para encontrar f(xi) y se regresara ese valor
    \item  \$normal(flotante media, flotante desv): Regresa un número escalar que pertenezca a la distribución normal dado los parámetros
    \item  \$poisson(flotante lambda): Regresa un número aleatorio de la distribución Poisson con la lambda dada
    \item  \$dexponencial(flotante beta): Regresa un número aleatorio de la distribución Exponencial correspondiente a la beta (o 1/Lambda) dada
    \item  \$dgeometrica(flotante exito): Te regresa un valor con la distribución geométrica con la probabilidad de exito dada
    \item  \$histograma(flotante x[], flotante rango): Genera un histograma a partir de los datos en el vector de x con un rango entre los datos de rango
    \item  \$diagramadecaja(flotante x[]): Genera un diagrama de caja y bigotes de los datos en en x
    \item  \$grafDispersion(flotante x[], flotante y[]): Genera un gráfico de dispersión con los valores de ‘x’ y ‘y’

\end{itemize}


\part{Ejemplos de Documentación en código}

Ejemplo de funciones documentadas en el compilador:
\scriptsize
\begin{minted}{python}
    # Func : getArrayData
    # Params : Una dirección Virtual addr
    # ret : La información de las dimensiones de un arreglo
    # Desc : Regresa la información de las dimensiones de un arreglo
    def getArrayData(addr):
        global dirfunc
        global currscope
        res = {}
    
        for var in dirfunc[currscope]['vartab'].values():
    
            if addr['address'] == var['address']:
                res['size'] = var['size']
                res['dimlen'] = var['dimlen']
                res['dims'] = var['dims']
                return res
        for var in dirfunc['global']['vartab'].values():
    
            if addr['address'] == var['address']:
    
                res['size'] = var['size']
                res['dimlen'] = var['dimlen']
                res['dims'] = var['dims']
                return res
        return res
    
   
\end{minted}
\scriptsize
\begin{minted}{python}
# Func : addConst
# Param : Una constante cte, un tipo tipo y la linea del token line
# Desc : Agrega una constante a la tabla de constante y le asigna una dirección 
# virtual de acuerdo al tipo de la constante.

def addConst(cte,tipo,line):
    # Rango de memoria constantes
    global constint
    global constfloat
    global constchar
    global constbool
    global conststring
    # Contadores constantes
    global cteintcount
    global ctefloatcount
    global ctecharcount
    global cteboolcount
    global ctestringcount
    # Maximos de variables
    global INTMAX
    global FLOATMAX
    global CHARMAX
    global BOOLMAX
    global STRINGMAX
    #Tabla const
    global ctetab
    global objctetab

    # Si la llave no es un string hacer una llave string
    if type(cte) is not str:
        ctkey = str(cte)
    else:
        ctkey = cte
    # Agregarla de acuerdo al tipo y con su dirección de memoria adecuada
    if tipo == 'entero':
        if cteintcount < INTMAX:
            ctetab[ctkey] = constint + cteintcount
            objctetab[ctetab[ctkey]] = cte
            cteintcount += 1
            return ctetab[ctkey]
        else:
            printerror(
                "Error de Semantica: sobrepaso el limite de constantes declaradas en la linea %r" % (line))
    elif tipo == 'flotante':
        if ctefloatcount < FLOATMAX:
            ctetab[ctkey] = constfloat + ctefloatcount
            objctetab[ctetab[ctkey]] = cte
            ctefloatcount += 1
            return ctetab[ctkey]
        else:
            printerror(
                "Error de Semantica: sobrepaso el limite de constantes declaradas en la linea %r" % (line))
    elif tipo == 'char':
        if ctecharcount < CHARMAX:
            ctetab[cte] = constchar + ctecharcount
            objctetab[ctetab[cte]] = cte
            ctecharcount += 1
            return ctetab[ctkey]
        else:

            printerror(
                "Error de Semantica: sobrepaso el limite de constantes declaradas en la linea %r" % (line))
    elif tipo == 'bool':
        if cteboolcount < BOOLMAX:
            ctetab[cte] = constbool + cteboolcount
            if cte == 'verdadero':
                objctetab[ctetab[cte]] = True
            else:
                objctetab[ctetab[cte]] = False
            cteboolcount += 1
            return ctetab[ctkey]
        else:
            printerror(
                "Error de Semantica: sobrepaso el limite de constantes declaradas en la linea %r" % (line))
    elif tipo == 'cadena':
        if ctestringcount < STRINGMAX:
            ctetab[ctkey] = conststring + ctestringcount
            objctetab[ctetab[ctkey]] = cte
            ctestringcount += 1
            return ctetab[ctkey]
        else:

            printerror(
                "Error de Semantica: sobrepaso el limite de constantes declaradas en la linea %r" % (line))

# Func : assignvirtualaddress
# Params : Una tabla de variables vartab, un scope addresscope y el número del linea de los tokens linenum
# Ret : Una tabla de variables actualizada con las direcciones de cada variable
# Desc : Asigna las direcciones virtuales a la tabla de variables
def assignvirtualaddress(vartab,addressscope,linenum):
    # Rangos globales de memoria
    global globalint
    global globalfloat
    global globalchar
    global globalbool
    global globalpoint
    # Direcciones locales
    global localint
    global localfloat
    global localchar
    global localbool
    global localpoint
    # Cant. Maximas de variables
    global INTMAX
    global FLOATMAX
    global CHARMAX
    global BOOLMAX
    global POINTAYMAX

    # Contadores globales
    global glbintcount
    global glbfloatcount
    global glbcharcount
    global glbboolcount
    #
    # Contadores de variables
    intcount = 0
    floatcount = 0
    charcount = 0
    boolcount = 0

    # Contadores de arreglos
    pointcount = 0

    if addressscope == 'global':
        for k in vartab.keys():
            if('dims' in vartab[k].keys()): # Asignacion de direcciones de arreglos
                incr = vartab[k]['size']
                isArr = True
            else:
                incr = 1
                isArr = False


            if(vartab[k]['tipo'] == 'entero'): #Asignar las direcciones enteras
                if (intcount < INTMAX):
                    vartab[k]['address'] = globalint + intcount
                    intcount += incr
                    if isArr:
                        addConst(vartab[k]['address'], 'entero', linenum)
                else:
                    printerror("Error de Semantica: sobrepaso el limite de variables declaradas en la linea %r" % (linenum))

            elif (vartab[k]['tipo'] == 'flotante'):#Asignar las direcciones flotantes
                if (floatcount < FLOATMAX):
                    vartab[k]['address'] = globalfloat + floatcount
                    floatcount += incr
                    if isArr:
                        addConst(vartab[k]['address'], 'entero', linenum)
                else:
                    printerror("Error de Semantica: sobrepaso el limite de variables declaradas en la linea %r" % (linenum))

            elif (vartab[k]['tipo'] == 'char'):#Asignar las direcciones char
                if (charcount < CHARMAX):
                    vartab[k]['address'] = globalchar + charcount
                    charcount += incr
                    if isArr:
                        addConst(vartab[k]['address'], 'entero', linenum)
                else:
                    printerror("Error de Semantica: sobrepaso el limite de variables declaradas en la linea %r" % (linenum))
            elif (vartab[k]['tipo'] == 'bool'):#Asignar las direcciones char
                if (boolcount < BOOLMAX):
                    vartab[k]['address'] = globalbool + boolcount
                    boolcount += incr
                else:
                    printerror("Error de Semantica: sobrepaso el limite de variables declaradas en la linea %r" % (linenum))
    elif addressscope == 'local':
        for k in vartab.keys():
            if('dims' in vartab[k].keys()): # Asignacion de direcciones de arreglos
                incr = vartab[k]['size']
                isArr = True
            else:
                incr = 1
                isArr = False

            if(vartab[k]['tipo'] == 'entero'): #Asignar las direcciones enteras
                if (intcount < INTMAX):
                    vartab[k]['address'] = localint + intcount
                    intcount += incr
                    if isArr:
                        addConst(vartab[k]['address'], 'entero', linenum)
                else:
                    printerror("Error de Semantica: sobrepaso el limite de variables declaradas en la linea %r" % (linenum))

            elif (vartab[k]['tipo'] == 'flotante'):#Asignar las direcciones flotantes
                if (floatcount < FLOATMAX):
                    vartab[k]['address'] = localfloat + floatcount
                    floatcount += incr
                    if isArr:
                        addConst(vartab[k]['address'], 'entero', linenum)
                else:
                    printerror("Error de Semantica: sobrepaso el limite de variables declaradas en la linea %r" % (linenum))

            elif (vartab[k]['tipo'] == 'char'):#Asignar las direcciones char
                if (charcount < CHARMAX):
                    vartab[k]['address'] = localchar + charcount
                    if isArr:
                        addConst(vartab[k]['address'], 'entero', linenum)
                    charcount += incr
                else:
                    printerror("Error de Semantica: sobrepaso el limite de variables declaradas en la linea %r" % (linenum))
            elif (vartab[k]['tipo'] == 'bool'):#Asignar las direcciones char
                if (boolcount < BOOLMAX):
                    vartab[k]['address'] = localbool + boolcount
                    if isArr:
                        addConst(vartab[k]['address'], 'entero', linenum)
                    boolcount += incr
                else:
                    printerror(
                    "Error de Semantica: sobrepaso el limite de variables declaradas en la linea %r" % (linenum)
                    )
    return vartab, intcount, floatcount, charcount, boolcount


\end{minted}

\newpage
Ejemplos de Documentación en la máquina virtual:
\scriptsize
\begin{minted}{python}
    
    
    # Func: storeinmem
    # Param: Una direccion virtual addrs, un valor val, y opcionalmente un flag indicando 
    # si guardar en la variable temporal era
    # Desc: Guardar en memoria el valor dado.
    def storeinmem(addrs,val, isera = False):
        scop, atype, aoff = getTypeAndOffset(addrs)
        if scop == 'global':
            if atype == 0: #entero
                globmem.mint[aoff] = val
            elif atype == 1: #float
                globmem.mfloat[aoff] = val
            elif atype == 2: #char
                globmem.mchar[aoff] = val
            elif atype == 3: #bool
                globmem.mbool[aoff] = val
            elif atype == 4: #pointer
                globmem.mpoint[aoff] = val
        elif scop == 'local':
            if atype == 0: #entero
                if isera:
                    eratemp[0].mint[aoff] = val
                else:
                    memstack[-1][0].mint[aoff] = val
            elif atype == 1: #float
                if isera:
                    eratemp[0].mfloat[aoff] = val
                else:
                    memstack[-1][0].mfloat[aoff] = val
            elif atype == 2: #char
                if isera:
                    eratemp[0].mchar[aoff] = val
                else:
                    memstack[-1][0].mchar[aoff] = val
            elif atype == 3: #bool
                if isera:
                    eratemp[0].mbool[aoff] = val
                else:
                    memstack[-1][0].mbool[aoff] = val
            elif atype == 4: #pointer
                if isera:
                    eratemp[0].mpoint[aoff] = val
                else:
                    memstack[-1][0].mpoint[aoff] = val
        elif scop == 'temp':
            if atype == 0: #entero
                if isera:
                    eratemp[1].mint[aoff] = val
                else:
                    memstack[-1][1].mint[aoff] = val
            elif atype == 1: #float
                if isera:
                    eratemp[1].mfloat[aoff] = val
                else:
                    memstack[-1][1].mfloat[aoff] = val
            elif atype == 2: #char
                if isera:
                    eratemp[1].mchar[aoff] = val
                else:
                    memstack[-1][1].mchar[aoff] = val
            elif atype == 3: #bool
                if isera:
                    eratemp[1].mbool[aoff] = val
                else:
                    memstack[-1][1].mbool[aoff] = val
            elif atype == 4: #pointer
                if isera:
                    if eratemp[1].mpoint[aoff] == None:
                        eratemp[1].mpoint[aoff] = val
                    else:
                        
                        storeinmem(eratemp[1].mpoint[aoff],val,isera)
                else:
                    
                    if memstack[-1][1].mpoint[aoff] == None:
                        memstack[-1][1].mpoint[aoff] = val
                        
                    else:
                        #val, _ = getexpoper(memstack[-1][1].mpoint[aoff])
                        
                        storeinmem(memstack[-1][1].mpoint[aoff],val)
                        
                        
    # Func: valtonum
    # params: Un valor val, y un tipo t
    # ret: Regresa el valor de v como el tipo t
    # Desc: Cambiar valor a numero si es necesario
    def valtonum(val,t):
        if t == 2:
            val = ord(val)
        elif t == 3:
            if val:
                val = 1
            else:
                val = 0
        return val
    
    
    # Func: reusePointer
    # Params: Una dirección virtual addrs y un valor 
    # Reescribir el valor del apuntador cuando es necesario
    def reusePointer(addrs,val):
        scop, atype, aoff = getTypeAndOffset(addrs)
        memstack[-1][1].mpoint[aoff] = val
    
    
    # Func : exeExpresion
    # Params : Un operador op, tres direcciones virtuals ladd, radd resadd
    # Desc : Ejecuta la acción del cuadruplo y la guarda en memoria
    def exeExpresion(op,ladd,radd,resadd):
        lop,lt = getexpoper(ladd)
        rop,rt = getexpoper(radd)
        if lop == None or rop == None:
            printerr(
            'Esta intentando realizar operaciónes con variables que no cuentan con un valor.'
            +
            '\nRevisa el códgo para asegurar que no haya alguna variable sin valor en alguna de tus operaciones'
            ,ip)
        lop = valtonum(lop,lt)
        rop = valtonum(rop,rt)
        aux = 0
        # Operadores ['*','/','+','-','>','<','>=','<=','!=','==','||','&&']
        if op == '*': 
            # Multiplicacion
            aux = lop * rop
    
        elif op == '/': 
            # Division
            aux = lop / rop
    
        elif op == '+': 
            #Suma 
            if rop == '':
                aux = abs(lop)
            else:
                aux = lop + rop
    
        elif op == '-': 
            #Resta
            if rop == '':
                aux = -1 * abs(lop)
            else:
                aux = lop - rop
    
        elif op == '>': 
            #Mayorque
            aux = lop > rop
    
        elif op == '<':
            #menor que
            aux = lop < rop
    
        elif op == '>=': 
            # mayor igual
            aux = lop >= rop
    
        elif op == '<=': 
            #menor igual
            aux = lop <= rop
    
        elif op == '==': 
            #igual (relacional)
            aux = lop == rop
    
        elif op == '!=': 
            #diferente
            aux = lop != rop
    
        elif op == '&&': 
            # and
            aux = lop and rop
    
        elif op == '||': 
            # or
            aux = lop or rop
        if type(resadd) is str: # Reescribir el pointer en caso de ser necesaria
            reusePointer(int(resadd[1:]),aux)
        else:
            storeinmem(resadd,aux)
\end{minted}

%\bibliographystyle{plain}
%\bibliography{bibtex_example}

\printindex

\end{document}
